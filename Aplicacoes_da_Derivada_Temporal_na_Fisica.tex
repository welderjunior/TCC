\documentclass[
	% -- opções da classe memoir --
	12pt,				% tamanho da fonte
	openright,			% capítulos começam em pág ímpar (insere página vazia caso preciso)
    twoside,			% para impressão em recto e verso. Oposto a oneside
	a4paper,			% tamanho do papel.
	% -- opções da classe abntex2 --
	chapter=TITLE,		% títulos de capítulos convertidos em letras maiúsculas
	%section=TITLE,		% títulos de seções convertidos em letras maiúsculas
	%subsection=TITLE,	% títulos de subseções convertidos em letras maiúsculas
	%subsubsection=TITLE,% títulos de subsubseções convertidos em letras maiúsculas
	% -- opções do pacote babel --
	english,			% idioma adicional para hifenização
	french,				% idioma adicional para hifenização
	spanish,			% idioma adicional para hifenização
	brazil				% o último idioma é o principal do documento
	]{abntex2}

% ---
% Pacotes básicos 
% ---
\usepackage{lmodern}			% Usa a fonte Latin Modern			
\usepackage[T1]{fontenc}		% Selecao de codigos de fonte.
\usepackage[utf8]{inputenc}		% Codificacao do documento (conversão automática dos acentos)
\usepackage{cancel}
\usepackage{lastpage}			% Usado pela Ficha catalográfica
\usepackage{indentfirst}		% Indenta o primeiro parágrafo de cada seção.
% Pacote para o uso de cores no documento
\usepackage{color}
% Pacote para a inclusão\manipulação de imagens
\usepackage{graphicx}
\usepackage{microtype} 			% para melhorias de justificação
% ---
\usepackage{tikz}
\usetikzlibrary{babel, quotes, angles}
\usepackage{tkz-base}
\usepackage{tkz-fct}

% ---
% Pacotes adicionais, usados apenas no âmbito do Modelo Canônico do abnteX2
% ---
\usepackage{lipsum}				% para geração de dummy text
% ---

% ---
% Pacotes para escrita matemática
% ---
\usepackage{amsmath}
\usepackage{amstext}
\usepackage{amssymb}	 % qed
\usepackage{amsthm}      % Teoremas

\newenvironment{amatrix}[1]{%
  \left(\begin{array}{@{}*{#1}{c}|c@{}}
}{%
  \end{array}\right)
}

% Cria um novo ambiente para Teorema
\newtheorem{teorema}{Teorema}
% Cria um novo ambiente para Definição
\newtheorem{definicao}{Definição}
% Cria um novo ambiente para Exemplo
\newtheorem{exemplo}{Exemplo}
% Cria um novo ambiente para Lema
\newtheorem{lema}{Lema}
%%%%%%%%%%%%%%%%%%%%%%%%%%%%%%%%%%%%%%%%%%%%%%%%%%%%%%%%%%%%%%%
% Recria o ambiente para demonstrações
% \begin{proof} [...] \end{proof}
% 			Observação: O ambiente foi recriado para adequação
%						das fontes com o restante do texto
%%%%%%%%%%%%%%%%%%%%%%%%%%%%%%%%%%%%%%%%%%%%%%%%%%%%%%%%%%%%%%%
\makeatletter
\renewenvironment{proof}[1][\proofname]{
	\par\pushQED{\qed}%
	\normalfont \topsep6\p@\@plus6\p@\relax
	\trivlist
	\item\relax
		{\itshape
			#1\@addpunct{.}}\hspace\labelsep\ignorespaces
}{%
	\popQED\endtrivlist\@endpefalse
}
\makeatother
%%%%%%%%%%%%%%%%%%%%%%%%%%%%%%%%%%%%%%%%%%%%%%%%%%%%%%%%%%%%%%%
% Cria o ambiente para solução de exemplos.
% \begin{solution} [...] \end{solution}
%%%%%%%%%%%%%%%%%%%%%%%%%%%%%%%%%%%%%%%%%%%%%%%%%%%%%%%%%%%%%%%
\makeatletter
\newenvironment{solution}{
	\begin{proof}[Solução]
}{%
	\end{proof}
}
\makeatother


% Reseta o numerador do ambiente Lema ao fim de todo
% capítulo. Assim, os Lemas serão indexados por capítulo
% e não no documento todo. Por exemplo, no Capítulo 1,
% tem-se, Lema 1.1, Lema 1.2, etc.
\numberwithin{lema}{chapter}
% Mesma função explicada acima para os ambientes de
% teorema, definição, exemplo, figure
\numberwithin{teorema}{chapter}
\numberwithin{definicao}{chapter}
\numberwithin{exemplo}{chapter}
\numberwithin{figure}{chapter}

% Pacotes do Tikz utilizados para desenho
\usepackage{tikz, tkz-euclide, tkz-fct}
\usetikzlibrary{babel}
%\usetkzobj{all}

% Pacote para a inclusão de códigos e para gerar
% "Syntax Highlight" (Coloração do Código)
\usepackage{listings}


% Define o nome de marcação para os Blocos de Código
\renewcommand{\lstlistingname}{Código}
% Define o nome de marcação para a Lista de Códigos
\renewcommand{\lstlistlistingname}{Lista de \lstlistingname s}
% Comandos utilizados para a criação e estilização
% da Lista de Códigos
\begingroup\makeatletter
\let\newcounter\@gobble\let\setcounter\@gobbletwo
	\globaldefs\@ne \let\c@loldepth\@ne
	\newlistof{listings}{lol}{\lstlistlistingname}
	\newlistentry{lstlisting}{lol}{0}
\endgroup
\renewcommand{\cftlstlistingaftersnum}{\hfill--\hfill}
%
% Comandos para a recriação do comando \lstlistoflistings,
% ou seja, para criar a Lista de Códigos
\let\oldlstlistoflistings\lstlistoflistings
\renewcommand{\lstlistoflistings}{%
	\begingroup%
	\let\oldnumberline\numberline%
	\renewcommand{\numberline}{\lstlistingname\space\oldnumberline}%
	\oldlstlistoflistings%
	\endgroup}
% Cria um novo estilo de formatação de código
% para a Linguagem de programação Python
% a ser utilizada no ambiente "lstlisting".
\lstdefinestyle{Python}{
	language={Python},
	basicstyle=\ttfamily\small,
	identifierstyle=\color{black},
	keywordstyle=\color{blue},
	commentstyle=\color{green},
	stringstyle=\color{red},
	extendedchars=true,
	showspaces=false,
	showstringspaces=false,
	numbers=none,
	breaklines=true,
	backgroundcolor=\color{yellow!20},
	breakautoindent=true,
	captionpos=b,
	xleftmargin=0pt,
	frame=none
}
% Cria estilo de código em preto-e-branco
% usado apenas para impressão (mais barata)
%\lstdefinestyle{Python}{
%	language={Python},
%	basicstyle=\ttfamily\small,
%	identifierstyle=\color{black},
%	keywordstyle=\color{black},
%	commentstyle=\color{black},
%	stringstyle=\color{black},
%	extendedchars=true,
%	showspaces=false,
%	showstringspaces=false,
%	numbers=none,
%	breaklines=true,
%	backgroundcolor=\color{gray!10},
%	breakautoindent=true,
%	captionpos=b,
%	xleftmargin=0pt,
%	frame=none
%}

\usepackage{subcaption}

\makeatletter
\renewcommand{\ABNTEXcaptionfontedelim}{,}
\renewcommand{\fonte}[2][\fontename]{%
	\M@gettitle{#2}%
	\memlegendinfo{#2}%
	\par
	\begingroup
		\@parboxrestore
		\if@minipage
			\@setminipage
		\fi
		\small
     	\configureseparator
		\@makecaption{\small #1}{\ignorespaces\small #2}\par
	\endgroup
}
\makeatother

\newcommand{\codigofonte}{
	\vspace{-0.4cm}
	\fonte{Elaborado pelo autor, 2019.}
	\vspace{0.4cm}
}

\makeatletter
\newcommand{\noEquationSpacing}[1]{
	{
		\setlength{\belowdisplayskip}{0pt} \setlength{\belowdisplayshortskip}{0pt}
		\setlength{\abovedisplayskip}{0pt} \setlength{\abovedisplayshortskip}{0pt}
		#1
	}
}
\makeatother

% ---
% Pacotes de citações
% ---

\usepackage[brazilian,hyperpageref]{backref}	 % Paginas com as citações na bibl
% Pacote abntex2cite para citações no padrão ABNT
\usepackage[alf]{abntex2cite}
% --- 
% CONFIGURAÇÕES DE PACOTES
% --- 



% Backref para a escrita do trabalho
%\renewcommand{\backrefpagesname}{Citado na(s) página(s):~}
%\renewcommand{\backref}{}
%\renewcommand*{\backrefalt}[4]{
%	\ifcase #1 %
%		Nenhuma citação no texto.%
%	\or
%		Citado na página #2.%
%	\else
%		Citado #1 vezes nas páginas #2.%
%	\fi}%
% ---

% Backref para produção
\renewcommand{\backrefpagesname}{}
\renewcommand{\backref}{}
\renewcommand*{\backrefalt}[4]{}

% ---
% Informações de dados para CAPA e FOLHA DE ROSTO
% ---
\titulo{\textbf{Aplicações da Derivada Temporal na Física}}
\autor{WELDER JUNIOR SILVA DE JESUS}
\local{Anápolis}
\data{2022}
\orientador{Prof. Dra. Selma Marques de Paiva}
%\coorientador{Titulação e Nome do coorientador}
\instituicao{%
Universidade Estadual de Goiás
  \par
Câmpus Central - Sede: Anápolis - CET
  \par
Curso de Matemática}
\tipotrabalho{Trabalho de Curso (Graduação)}
% O preambulo deve conter o tipo do trabalho, o objetivo, 
% o nome da instituição e a área de concentração 
\preambulo{Trabalho de Curso (TC) apresentado à Coordenação Setorial do Curso de Matemática, como parte dos requisitos para obtenção do título de Graduado no Curso de Matemática da Universidade Estadual de Goiás, sob a orientação da Professora Dra. Selma Marques de Paiva.}
% ---


% ---
% Configurações de aparência do PDF final

% alterando o aspecto da cor azul
\definecolor{blue}{RGB}{41,5,195}
% informações do PDF
\makeatletter
\hypersetup{
     	%pagebackref=true,
		pdftitle={\@title}, 
		pdfauthor={\@author},
    	pdfsubject={\imprimirpreambulo},
	    pdfcreator={LaTeX with abnTeX2},
		pdfkeywords={abnt}{latex}{abntex}{abntex2}{trabalho acadêmico}, 
		colorlinks=false,       		% false: boxed links; true: colored links
    	linkcolor=blue,          	% color of internal links
    	citecolor=blue,        		% color of links to bibliography
    	filecolor=magenta,      		% color of file links
		urlcolor=blue,
		bookmarksdepth=4
}
\makeatother
% --- 

% Define o tamanho da indentação do paragrafo
\setlength{\parindent}{1.5cm}
% Define o espaçamento entre um parágrafo e outro
\setlength{\parskip}{0.2cm}

% Customizações para o Curso de Matemática do Campus CET
% da Universidade Estadual de Goiás.
\usepackage{abntex2matematicaCCET}

% Pacote genealogytree. Usado apenas para os símbolos de
% nascimento e morte, usados nessas datas no capítulo
% sobre história.
\usepackage{genealogytree}
% Cria um comando utilizado para inserir a data de nascimento
% e morte como nota de rodapé.
\newcommand{\bdDate}[2]{\footnote{\gtrsymBorn\text{ }#1 - \gtrsymDied\text{ }#2}}


\usepackage{mathptmx}
\renewcommand{\ABNTEXchapterfont}{\normalfont}

% 
\usepackage{pdfpages}

% pacotes utilizados para a configuração do índice remissivo
\usepackage{imakeidx}
\usepackage{hyperref}

% Altera as configurações do index (página de índice remissivo)
% alterando o estilo da página para o estilo 'abntchapfirst' do
% abntex, fazendo com que o número de página vá para o local
% correto.
\indexsetup{
  othercode={%
    \thispagestyle{abntchapfirst}
  }
}
%
% ---
% compila o indice
% ---
\makeindex
\makeindex[name=remissivo, title={Índice Remissivo}, options=-s indexstyle, columns=2]

\begin{document}
% Seleciona o idioma do documento (conforme pacotes do babel)
\selectlanguage{brazil}
% Retira espaço extra obsoleto entre as frases.
\frenchspacing


% ----------------------------------------------------------
% ELEMENTOS PRÉ-TEXTUAIS
% ----------------------------------------------------------
\pretextual
% ---
% Capa
% ---
\imprimircapa
% ---
% ---
% Folha de rosto
% (o * indica que haverá a ficha bibliográfica)
% ---
\imprimirfolhaderosto*
% ---

% ---
% Inserir a ficha bibliografica (Catalográfica)
% ---

% na pasta "arquivos" coloque o documento, em formato PDF, com % a ficha catalográfica Mdefinitiva após a defesa do trabalho.
% Para este caso o documento está nomeado como
% "ficha_catalografica".
%
 \begin{fichacatalografica}
     \includepdf[pages=1]{arquivos/ficha_catalografica.pdf}
 \end{fichacatalografica}


% ---
% Inserir folha Termo de autorização para disponibilização de monografias digitais no banco de dados do Câmpus CET.
% ---

% A posição deste item na monografia poderá ser corrigida após consulta à bibliotecária.
% Inserir o documento, em formato PDF, contendo o termo de autorização para disponibilização de monografias digitais após ser preenchido e devidadamente assinado.

\includepdf{arquivos/Autorizacao_disponibilizar_tcc_Biblioteca.pdf}
\cleardoublepage




% ---
% Inserir folha de aprovação
% ---

% na pasta "arquivos" coloque o documento, em formato PDF, com
% a folha de aprovação definitiva após a defesa do trabalho e
% devidas assinaturas.
% Para este caso o documento está nomeado como
% "folha_aprovacao".

\begin{folhadeaprovacao}
\includepdf[page=-]{arquivos/folha_aprovacao.pdf}
\cleardoublepage
\end{folhadeaprovacao}

% ---


% ---
% Dedicatória
% ---

% ---
% Agradecimentos
% ---
\begin{agradecimentos}[AGRADECIMENTOS]
	
No decorrer dos últimos 4 anos da minha graduação, tive o privilégio de conhecer e conviver com grandes amigos que dispuseram-se a me ajudar neste trajeto. Antes de agradecê-los, primeiramente quero deixar os meus agradecimentos a minha família, que esteve presente e apoiando as minhas escolhas.

Deixo os meus agradecimentos à minha orientadora Prof. Dra. Selma Marques de Paiva, por conceder a oportunidade de ser seu orientando. Agradeço por toda orientação, dicas e ensinamentos que foram transmitidos no decorrer deste trabalho, que propiciou na construção do mesmo.

Sou grato à Karol e José Eduardo, que são grandes amigos e compartilham da mesma orientadora que eu. Agradeço pelas dicas e experiências que foram transmitidas por ambos, que proporcionou uma melhoria neste trabalho.

Também agradeço aos meus outros grandes amigos, Aline, Daniel, Lillian, Lucas, Matheus, Nicolas, Salomão e também à aqueles que mesmo não sendo citados, contribuíram nesta grande caminhada.

Por fim, mas de forma alguma menos importante, dedico meus agradecimentos a todos os professores do curso que compartilharam de seus conhecimentos e experiências nesses últimos anos.

\end{agradecimentos}
% ---

% ---
% Epígrafe
% ---
\begin{epigrafe}
    \vspace*{\fill}
	\begin{flushright}
		\textit{"Alegações extraordinárias exigem evidências extraordinárias." \\
		(Carl Sagan)}
	\end{flushright}
\end{epigrafe}
% ---

% ---
% RESUMOS
% ---

% resumo em português
\setlength{\absparsep}{18pt} % ajusta o espaçamento dos parágrafos do resumo
\begin{resumo}[RESUMO]
	
O Cálculo é um importante ramo da matemática que possui diversas aplicações nas mais variadas áreas, sendo uma delas a Física. Com o objetivo de mostrar algumas aplicações da taxa de variação temporal que estão presentes nesta área, inicialmente será realizado um breve relato histórico percorrendo através das grandes mentes que desenvolveram o Cálculo. Em seguida, será introduzido o Cálculo Diferencial com alguns conceitos importantes que compõem o mesmo, sendo eles a taxa de variação média, a taxa de variação instantânea e as suas respectivas interpretações geométricas. Ainda abordando o Cálculo Diferencial, algumas de suas regras de derivação serão expostas com a finalidade de facilitar a derivação das futuras funções implementadas. Logo após isso, as taxas de variações temporais serão exploradas, sendo elas, a velocidade e aceleração instantânea. Contudo, antes de explorar tais taxas de variações, as definições de velocidade e aceleração serão aprofundadas formalmente. Por fim, chegamos ao objetivo deste trabalho de mostrar algumas das aplicações da taxa de variação temporal presentes na Física.

\textbf{Palavras-chave}: Cálculo Diferencial, Derivada Temporal, Taxa de Variação.

\end{resumo}

% ---
% inserir lista de ilustrações
% ---
\pdfbookmark[0]{\listfigurename}{lof}
\listoffigures*
\cleardoublepage
% ---

% ---
% inserir lista de tabelas
% ---
\pdfbookmark[0]{\listtablename}{lot}
\listoftables*
\cleardoublepage
% ---

% ---
% inserir lista de abreviaturas e siglas
% ---
%\begin{siglas}
%  \item[ABNT] Associação Brasileira de Normas Técnicas
%  \item[abnTeX] ABsurdas Normas para TeX
%\end{siglas}
% ---

% ---
% inserir lista de símbolos
% ---
%\begin{simbolos}
%  \item[$ \Gamma $] Letra grega Gama
%  \item[$ \Lambda $] Lambda
%  \item[$ \zeta $] Letra grega minúscula zeta
%  \item[$ \in $] Pertence
%\end{simbolos}
% ---


% ---
% inserir o sumario
% ---
\pdfbookmark[0]{\contentsname}{toc}
\tableofcontents*
\cleardoublepage
% ---

% ----------------------------------------------------------
% ELEMENTOS TEXTUAIS
% ----------------------------------------------------------
\textual
\newtheorem{prop}{Proposição}[]
% ----------------------------------------------------------
% Introdução (exemplo de capítulo sem numeração, mas presente no Sumário)
% ----------------------------------------------------------
\chapter*[INTRODUÇÃO]{INTRODUÇÃO}
\addcontentsline{toc}{chapter}{INTRODUÇÃO}
\thispagestyle{empty}

Em meados do século XVII, como descrito por \citeonline{bardi}, o Cálculo estava sendo desenvolvido independentemente por dois pensadores $-$ Sir Isaac Newton e Gottfried Leibniz. Mesmo após poucos séculos do desenvolvimento, o Cálculo diferencial apresenta uma vasta gama de aplicações nas mais diversas áreas. Pensando nisso, este trabalho visa mostrar algumas aplicações da taxa de variação temporal, tendo como escopo, a área da Física que possui diversas aplicações interessantes.

Perante este objetivo, a elaboração do trabalho deu-se através de um levantamento bibliográfico em diversos livros, nos quais, os autores abordam os respectivos temas descritos no decorrer deste trabalho. Sendo dividido em três Capítulos, este trabalho contempla diversas informações e conceitos referentes ao Cálculo Diferencial e, supostamente, à velocidade e aceleração instantâneas que são taxas de variações temporais.

O primeiro Capítulo, retrata um breve contexto histórico relacionado com a criação e desenvolvimento do Cálculo, explorando um pouco a vida dos dois desenvolvedores independentes deste magnífico ramo da matemática. Além disso, discorre-se a respeito da Guerra do Cálculo, um acontecimento histórico tendo como principais envolvidos Newton e Leibniz, os respectivos indivíduos tidos como criadores do Cálculo.

O segundo Capítulo é composto por três seções, onde a primeira seção aborda principalmente a definição da taxa de variação média, necessária para compreender o que é uma taxa de variação instantânea, também conhecida como derivada. A segunda seção aborda exatamente esta taxa de variação instantânea, sendo ela essencial para a aplicação da taxa de variação temporal na Física. Por fim, a terceira seção é composta pelas regras de derivação, como o próprio nome induz, são regras que auxiliam na derivação de funções.

O terceiro Capítulo também é divido em três seções. Vale salientar que, a velocidade e aceleração instantâneas são taxa de variações temporais, e elas são as principais aplicações que estão presentes neste trabalho. Deste modo, a primeira seção terá como foco a definição formal da velocidade e aceleração. Na sequência, a segunda seção é composta pela velocidade e aceleração instantânea. Mediante toda esta abordagem, por fim chegamos a terceira seção que engloba o objetivo deste trabalho, que é o de apresentar algumas aplicações da taxa de variação temporal na Física.

\chapter{UM BREVE RELATO HISTÓRICO}
\label{capitulo_1} % TÍTULO EM CAIXA ALTA
\thispagestyle{empty}
%\addcontentsline{toc}{chapter}{Contexto Histórico}

Tem-se o intuito de abordar neste Capítulo como se deu o desenvolvimento do Cálculo, dando ênfase à história dos principais estudiosos do assunto que foram os precursores do referido tema. Também será feito um breve levantamento histórico acerca da denominada Guerra do Cálculo. Além disso, serão retratadas algumas das principais descobertas dos precursores do Cálculo.

Tal abordagem tem por finalidade mostrar o contexto histórico permeado ao desenvolvimento do Cálculo, com o intuito de despertar a curiosidade de futuros leitores a respeito do assunto.

\section{A Guerra do Cálculo}
Segundo \citeonline{bardi}, o Cálculo, que é um importante ramo da matemática, teria sido descoberto e desenvolvido por duas mentes independentes em meados do século XVII, primeiramente por Sir Isaac Newton (1643-1726) e alguns anos depois por Gottfried Leibniz (1646-1716). Para \citeonline{bardi}, tanto Newton como Leibniz descobriram o Cálculo de modo independente. Sabe-se, entretanto, que na época do descobrimento do Cálculo houve um conflito entre esses dois intelectuais, em que cada um disputava pela a autoria da descoberta do Cálculo. Sendo assim, foi travado um grande atrito entre Newton e Leibniz que durou vários anos e ficou conhecido como a Guerra do Cálculo.

A fim de explorar mais esse contexto histórico, iremos abordar algumas datas importantes para o contexto em questão. Na época do nascimento de Isaac Newton o calendário em vigor era o juliano. Segundo este calendário, Newton teria nascido dia 25 de dezembro de 1642, mesmo ano em que Galileu\footnote{Segundo \citeonline{howard}, Galileu Galilei (1564 - 1642) foi um importante astrônomo do século XVII, nasceu em Pisa, Itália. Ele foi responsável por grandes contribuições na área da ciência, sendo algumas delas, a invenção do microscópio moderno, notou a natureza parabólica da trajetória de um projétil no vácuo, fundou a mecânica dos corpos em queda livre e lançou os fundamentos da dinâmica em geral.} faleceu. Já pelo calendário gregoriano, que é o mais utilizado atualmente, datou-se o seu nascimento em 4 de janeiro de 1643. De acordo com \citeonline{pensadores}, Sir Isaac Newton (Figura \ref{fig_newton}) teria nascido no condado de Lincolnshire, Inglaterra. Segundo \citeonline{bardi}, Newton veio de uma família humilde e de poucos estudos, ele teria herdado o nome de seu pai que morreu aos 37 anos, poucos meses antes dele nascer. Sendo assim, ele não teve a oportunidade de conhecer seu pai, mas apenas sua mãe, Hannah Ayscough Newton, que ficou sendo uma viúva grávida e que, em 1645, casa-se com Barnabas Smith, um clérigo educado em Oxford. A partir deste novo matrimónio, Newton teve três meio-irmãos nomeados como Benjamin, Mary e Hannah.

\begin{figure}[h]
	\caption{Isaac Newton}
	\centering
	\includegraphics[scale=1.1]{./Imagens/Criadores/Newton.jpg}
	\label{fig_newton}
	\legend{\ABNTEXfontereduzida Fonte: BOYER, Carl Benjamin (1974, p.288)}
\end{figure}

Ainda de acordo com \citeonline{pensadores}, Newton foi um dos maiores cientistas dos últimos tempos, aos 18 anos de idade ele ingressou na Universidade de Cambridge e em 1668, conseguiu completar o seu doutorado. Conforme \citeonline{boyer} e \citeonline{ian}, Newton foi o autor de grandes livros, como \textit{Princípios Matemáticos de Filosofia Natural} e \textit{Óptica} que são considerados até hoje clássicos da Física. As grandes descobertas que ele fez durante a sua vida foram: o Teorema Binomial que o permitiu desenvolver o Cálculo, a formulação da Lei da Gravitação Universal, as três leis do movimento  - Lei da Inércia, Lei da Superposição de Forças ou Princípio Fundamental da Dinâmica e a Lei da Ação e Reação - e a natureza das cores.

O descobrimento do Cálculo feito por Newton, segundo \citeonline{bardi}, foi entre os anos de 1665 e 1666, quando ele era estudante da Universidade de Cambridge e estava em retiro na sua propriedade rural para escapar de uma epidemia de peste bubônica. Ao longo desses dois anos de retiro em sua propriedade rural, Newton foi capaz de descobrir o Cálculo, onde o denominou como Método de Fluxos e Fluentes. Inicialmente, Newton tinha a intenção de publicar os seus trabalhos sobre Cálculo e Ótica ao mesmo tempo, mas não publicou devido aos acontecimentos que ocorreram em 1672, com a publicação da sua Teoria das Cores\footnote{De acordo com \citeonline{bardi}, a Teoria das Cores de Newton propôs que a luz branca não era pura como presumia-se na época, que ela era a ausência de cor, mas sim uma mistura heterogênea de diferentes cores, as cores do arco-íris, misturadas em uma proporção adequada tornando-a branca. Fato que não foi aceitado pelos cientistas, principalmente por Robert Hooke, maior autoridade em relação à ótica na época e membro da Royal Society.}, que foi criticada negativamente e, desde então, Newton jurou não publicar mais nada. Embora a publicação de seu trabalho foi feita anos após o seu desenvolvimento, Newton compartilhava algumas cópias privadas com seus amigos.

Em 1º de julho de 1646, ainda durante a Guerra dos Trinta Anos (1618 - 1648) que ocorria na Europa, de acordo com \citeonline{bardi}, nasce Gottfried Leibniz (Figura \ref{fig_leibniz}) na cidade de Leipzig, situada na Alemanha, filho de Friedrich Leibniz e Catarina Schmuck. Em 1652, quando Leibniz tinha apenas 6 anos, seu pai faleceu. Aos 8 anos Leibniz teve acesso a uma vasta gama de livros que estavam contidos na biblioteca de seu falecido pai, dentre os vários livros presentes na biblioteca, estavam os livros de Platão, Heródoto, clássicos latinos, etc. Sendo assim, desde muito cedo, Leibniz teve a oportunidade de mergulhar nos livros em busca de conhecimento.

\begin{figure}[h]
	\caption{Gottfried Leibniz}
	\centering
	\includegraphics[scale=0.45]{./Imagens/Criadores/Leibniz.jpg}
	\label{fig_leibniz}
	\legend{\ABNTEXfontereduzida Fonte: BOYER, Carl Benjamin (1974, p.293)}
\end{figure}

Aos 15 anos de idade, conforme retratado por \citeonline{pensadores}, Leibniz ingressa na Universidade de Leipzig, Alemanha, onde estudou sobre teologia, direito, filosofia e matemática e aos 20 anos já estava preparado para receber o título de doutor, que fora-lhe negado em virtude de sua pouca idade. Devido a essa circunstância, Leibniz deixou a Universidade de Leipzig e foi receber o título de doutor na Universidade de Altdorf, na Suíça. Sua principal contribuição para a área da matemática foi a descoberta do Cálculo independentemente de Newton, onde criou uma notação que ainda é usada atualmente. 

De acordo com \citeonline{bardi}, o descobrimento do Cálculo feito por Leibniz foi entre os anos de 1672 e 1676, quando ele estava em Paris. Embora, Leibniz fosse um advogado, ele conseguiu descobrir o Cálculo e criou um método totalmente original com símbolos novos. Após dez anos do descobrimento do Cálculo feito por Leibniz, ele aperfeiçoou o seu trabalho e publicou dois artigos datados em 1684 e 1686 e ficou conhecido como o inventor do Cálculo.

A Guerra do Cálculo foi um acontecimento histórico que ocorreu após as publicações de Leibniz relacionadas ao Cálculo, onde Newton acreditando que Leibniz tinha plagiado a sua ideia em relação ao Cálculo, decidiu lutar pela autoria do mesmo.

\begin{citacao}
Leibniz havia visto alguma coisa do trabalho inicial e inédito de Newton, o que foi o suficiente para dar a entender a este que Leibniz era um ladrão. Uma vez convencido disso, Newton passou decisivamente à ofensiva e utilizou sua reputação com grande efeito \cite[p. 12]{bardi}.
\end{citacao}

Desta forma, Newton estava empenhado em reivindicar a descoberta do Cálculo e decidiu partir para o ataque ofensivo publicando matérias denegrindo a imagem de Leibniz. Diante disso, Leibniz sem deixar as ofensas sem respostas, escreveu vários artigos defendendo a sua autoria do Cálculo  e consequentemente atacando Newton de volta. 

A Guerra do Cálculo durou vários anos sem que nenhuma parte abrisse mão da autoria de sua descoberta. No auge desta Guerra, ambos realizavam ataques tanto abertamente como em segredo, e mesmo após a morte de Leibniz em 1716, Newton continuou publicando artigos em sua defesa a fim de reivindicar a autoria do Cálculo. Desta forma, Newton conseguiu mostrar que descobriu o Cálculo antes de Leibniz, mas mesmo diante do fato dele ter descoberto o Cálculo cronologicamente primeiro, Leibniz descobriu o Cálculo independentemente e o desenvolveu mais do que Newton.
  
\chapter{TAXAS DE VARIAÇÃO E DERIVAÇÃO}
\label{capitulo_2}
\thispagestyle{empty}

Este Capítulo tem a finalidade de apresentar o Cálculo Diferencial, abordando as definições de uma taxa de variação média, taxa de variação instantânea e derivada, com suas respectivas interpretações geométricas. Será apresentado também as regras de derivação que compõem o Cálculo Diferencial. Para apresentar as definições de uma maneira mais rica teremos como base os autores \citeonline{andre}, \citeonline{ayrton}, \citeonline{geraldo}, \citeonline{guidorizzi}, \citeonline{leithold}, \citeonline{pedro}, \citeonline{thomas} e \citeonline{stewart}.

\section{Taxa de variação média}
\label{secao_1}
O Cálculo Diferencial é um importante ramo da matemática que estuda a taxa de variação que há em $y$ em função de $x$, isto é, a variação que ocorre em uma função $y=f(x)$ em um determinado intervalo. Sendo assim, há uma variação que ocorre nas variáveis $x$ e $y$, que são representadas por $\Delta x$ e $\Delta y$, respectivamente. Portando, temos $x$ variando de $x_0$ para $x_1$, acarretando em $\Delta x = x_1 - x_0$ e, de modo análogo, o mesmo acontece com a variação que há em $y$, ou seja, temos $f(x_0)$ variando para $f(x_1)$ acarretando em $\Delta y = f(x_1) - f(x_0)$, o que pode ser observado no exemplo da Figura \ref{funcao_1}.

%Cria um ambiente para colocar uma figura
\begin{figure}[h]
	%Título da figura
	\caption{Função $y = f(x)$}
	%Centraliza a figura
	\centering
		%Cria um ambiente para desenhar gráficos
		\begin{tikzpicture}
			%Cria um limite máximo e mínimo para os eixos x e y
			\tkzInit[xmin = -4, xmax = 4, ymin = -1, ymax = 6]
			%Desenha os eixos x e y
			\tkzDrawX[label={$x$}, noticks]
			\tkzDrawY[label={$y$}, noticks]
			
			%Comando para desenhar uma função com o seu respectivo domínio			
			\tkzFct[domain=-10:10]{x**2}					
			%Cria pontos na função
			\tkzDefPoints{
					1/1/P_0,
					2/4/P_1,
					1/0/x_0,
					2/0/x_1,
					-0.5/1.3/f1,
					-0.5/3.7/v1,
					0/1/f,
					0/4/v,
					1.2/-0.3/f2,
					1.2/-0.1/v2,
					1.8/-0.3/f3,
					1.8/-0.1/v3}
				
			\tkzDrawSegment[dashed](P_0, x_0)
			\tkzDrawSegment[dashed](P_0, f)	
			\tkzDrawSegment[dashed](P_1, v)
			\tkzDrawSegment[dashed](P_1, x_1)
			
			%Desenha os pontos
			\tkzDrawPoints(P_0, P_1, x_0, x_1, f, v)
			\tkzLabelPoints[below](x_0, x_1)
			\tkzLabelPoints[right](P_0, P_1)
			%Cria algumas caixas de texto
			\tkzText(-0.5,1){$f(x_0)$}
			\tkzText(-0.5,4){$f(x_1)$}
			\tkzText(-1, 2.5){$\Delta y$}
			\tkzText(1.5, -0.5){$\Delta x$}
			\tkzText(-3, 5){$y = f(x)$}
			%Cria alguns segmentos 
			\tkzDrawSegment(f1, v1)
			\tkzDrawSegment(f2, v2)
			\tkzDrawSegment(f3, v3)
			\tkzFct[domain=-0.6:-0.4]{x = 1.3}
			\tkzFct[domain=-0.6:-0.4]{x = 3.7}
			\tkzFct[domain=1.2:1.8]{x = -0.2}

			%reta tangente
			%\tkzDrawTangentLine[kr = 2.5, kl = 1 , draw](1)
			
		\end{tikzpicture}
	\label{funcao_1}
	\legend{\ABNTEXfontereduzida Fonte: Elaborada pelo autor, 2021.}
\end{figure}

A taxa de variação de $\Delta y$ em relação a $\Delta x$ é o quociente $\dfrac{\Delta y}{\Delta x}$ que é denominado como taxa de variação média de $y$ em relação a $x$ no intervalo [$x_0, x_1$].

Como a taxa de variação média é $\dfrac{\Delta y}{\Delta x}$, tem-se:
\begin{equation} 
	\label{taxa_med}	
	\frac{\Delta y}{\Delta x} = \frac{f(x_1) - f(x_0)}{x_1 - x_0}	
\end{equation}

Analisando a Figura \ref{funcao_2}, é possível traçar uma reta $s$ pelos pontos $P_0$ e $P_1$ do gráfico da Figura \ref{funcao_1}, obtendo-se assim uma reta que intercepta a função em dois pontos, ou seja, a reta $s$ é uma reta secante em y = f(x). A partir de um ponto $A(x_1, f(x_0))$ é possível formar um triângulo $P_0AP_1$ com os segmentos de reta $\overline{P_0A}$, $\overline{AP_1}$ e $\overline{P_0P_1}$.


\begin{figure}[h]
	%Título da figura
	\caption{Função $y = f(x)$ com a reta secante $s$}
	%Centraliza a figura
	\centering
	%Cria um ambiente para desenhar gráficos
	\begin{tikzpicture}
		%Cria um limite máximo e mínimo para os eixos x e y
		\tkzInit[xmin = -4, xmax = 4, ymin = -1, ymax = 6]
		%Desenha os eixos x e y
		\tkzDrawX[label={$x$}, noticks]
		\tkzDrawY[label={$y$}, noticks]
		
		%Comando para desenhar uma função com o seu respectivo domínio			
		\tkzFct[domain=-10:10]{x**2}	
		%reta tangente
		%\tkzDrawTangentLine[color = purple, kr = 2.5, kl = 1 , draw](1)	
		%\tkzText(4, 6){$t$}			
		%Cria pontos na função
		\tkzDefPoints{
			1/1/P_0,
			2/4/P_1,
			1/0/x_0,
			2/0/x_1,
			-0.5/1.3/f1,
			-0.5/3.7/v1,
			0/1/f,
			0/4/v,
			1.2/-0.3/f2,
			1.2/-0.1/v2,
			1.8/-0.3/f3,
			1.8/-0.1/v3}
		
		\tkzDrawSegment[dashed](P_0, x_0)
		\tkzDrawSegment[dashed](P_0, f)	
		\tkzDrawSegment[dashed](P_1, v)
		\tkzDrawSegment[dashed](P_1, x_1)
		
		%Desenha os pontos
		\tkzDrawPoints(P_0, P_1, x_0, x_1, f, v)
		\tkzLabelPoints[below](x_0, x_1)
		\tkzLabelPoints[right](P_1)
		\tkzLabelPoints[above left](P_0)
		%Cria algumas caixas de texto
		\tkzText(-0.5,1){$f(x_0)$}
		\tkzText(-0.5,4){$f(x_1)$}
		\tkzText(-1, 2.5){$\Delta y$}
		\tkzText(1.5, -0.5){$\Delta x$}
		\tkzText(-3, 5){$y = f(x)$}
		\tkzText(1.55, 1.3){$\alpha$}
		%Cria alguns segmentos 
		\tkzDrawSegment(f1, v1)
		\tkzDrawSegment(f2, v2)
		\tkzDrawSegment(f3, v3)
		\tkzFct[domain=-0.6:-0.4]{x = 1.3}
		\tkzFct[domain=-0.6:-0.4]{x = 3.7}
		\tkzFct[domain=1.2:1.8]{x = -0.2}
		%reta secante
		\tkzFct[color = red, domain=-1:6]{3*x - 2}
		\tkzText(3, 6){$s$}
		
		%triângulo
		\tkzDefPoints{2/1/A}
		\tkzDrawPoints(A)
		\tkzLabelPoints[right](A)
		
		%desenha o angulo
		\tkzDefPoints{
			2/4/D,
			1/1/E,
			2/1/F}	
		\tkzDrawPolygon(F, E, D)
		%angulo
		\draw(1.4, 1) arc [radius = 0.5cm, start angle=0, end angle=62];
		
		
		
	\end{tikzpicture}
	\label{funcao_2}
	\legend{\ABNTEXfontereduzida Fonte: Elaborada pelo autor, 2021.}
\end{figure}

Segundo \citeonline{stewart}, a definição da tangente de um ângulo $\theta$ é
\begin{equation} 
	\label{def_tan}	
	\tan \theta = \frac{op}{adj},
\end{equation}
onde $op$ é cateto oposto e $adj$ é o cateto adjacente, ambos em relação ao ângulo $\theta$. Sendo assim, analisando o triângulo formado na Figura \ref{funcao_2}, temos que a tangente do seu ângulo $\alpha$, de acordo com (\ref{def_tan}), é
\begin{equation} 
	\label{tangete_p_0}	
	\tan \alpha = \frac{f(x_1) - f(x_0)}{x_1 - x_0}	
\end{equation}

Logo, (\ref{tangete_p_0}) é a inclinação da reta secante $s$. Sendo assim, note que substituindo (\ref{tangete_p_0}) em (\ref{taxa_med}) temos:
\begin{equation}
	\label{taxa_tan}
	\frac{\Delta y}{\Delta x} = \tan \alpha
\end{equation}

Portanto, a interpretação geométrica da taxa de variação média é a inclinação da reta secante $s$ em uma função $y = f(x)$.

Podemos citar, a título de exemplo, a taxa de crescimento populacional na área da biologia, onde um aumento na população das bactérias depende do tempo, ou seja, em um instante $t_0$ existe uma população inicial que, no decorrer do tempo, sofrerá um acréscimo ou decréscimo. Ainda no mesmo exemplo, admita $y=f(t)$ como sendo a função que relaciona a população de bactérias em relação ao tempo, e o intervalo de tempo $t_0$ até $t_1$ sendo o tempo inicial e tempo final, respectivamente. A partir dos conceitos e definições apresentados na seção \ref{secao_1}, temos que a taxa de variação média neste contexto é exatamente o crescimento médio da população de bactérias no intervalo de tempo $t_0$ até $t_1$.

Outro exemplo seria na área da economia, uma função que relaciona o custo total de produção com a quantidade produzida de um determinado produto, isto é, se a quantidade produzida aumentar o custo total será maior e, se a quantidade produzida for menor acarreta em um custo total menor. Dessa forma, o custo total depende da quantidade produzida. Aplicando a taxa de variação média neste contexto, obtemos uma variação entre o custo total de produção em relação a quantidade produzida, logo têm-se o custo médio da produção de cada unidade.

%seção 2
\section{Taxa de variação instantânea}
\label{secao_2}

Em diversas áreas do conhecimento, existem problemas que são resolvidos a partir da análise do comportamento de funções. Esta análise, por sua vez, se dá a partir do Cálculo Diferencial. O estudo do movimento na Física também tem suas aplicações utilizando o Cálculo Diferencial, como por exemplo, a função que relaciona a posição de um objeto em relação ao tempo e outras aplicações que também serão abordadas no Capítulo \ref{capitulo_3}. 

As áreas do conhecimento que citamos até então, utilizam funções com grandezas que dependem do tempo para analisar o comportamento de determinada situação, mas isso não significa necessariamente que o Cálculo Diferencial aborda apenas problemas que dependem da variável tempo. Sendo assim, é possível analisar funções que no seu contexto apresentam grandezas que dependem de outras variáveis.

Na análise do comportamento de funções é interessante, e, às vezes necessário, em determinadas situações, analisar o que acontece em determinado instante $t$ ou momento do intervalo de alguma grandeza. Nos exemplos anteriores, temos a Taxa de Crescimento Populacional, sendo possível calcular a variação média da população de bactérias. Mas em determinadas situações pode ser necessário saber qual foi a variação da população em determinado instante. 

No ramo da economia, procura-se encontrar o Custo Marginal, ou seja, o custo da produção de uma unidade a mais. Sendo assim, para realizar essa análise é utilizada a taxa de variação instantânea que é um valor aproximado do Custo Marginal real.

Antes de definir a taxa de variação instantânea, abordaremos primeiramente, o problema de encontrar uma reta tangente\footnote{"TANGENTE, adj. Que toca; s. f. linha que toca outra linha ou superfície num só ponto; (Gem.) - trigonométrica: relação entre o seno e o cosseno de um ângulo."\cite[p. 742]{dicionario}} a uma determinada curva. O processo para encontrar uma reta tangente a um círculo é de fato trivial, segundo a definição da reta tangente (\ref{tangente}) de Euclides.

\begin{definicao}
	\label{tangente}
	\cite{euclides} Uma reta que, tocando o círculo e, sendo prolongada, não o corta, é
	dita ser tangente ao círculo.
\end{definicao}

O problema aparece ao tentar encontrar uma reta tangente a uma curva em geral, pois, essa definição não é apropriada para todos os casos, como pode ser visto nas Figuras \ref{exemplo_1} e \ref{exemplo_2}.

\begin{figure}[!h]
	\centering
	\begin{minipage}[!]{0.45\linewidth}
		\centering
		\caption{Exemplo 1}
		\label{exemplo_1}
			\begin{tikzpicture}
				\tkzInit[xmin = -3, xmax = 3, ymin = -1, ymax = 4]
				\tkzDrawX[label={$x$}, noticks]
				\tkzDrawY[label={$y$}, noticks]
				\tkzFct[domain=-3:3]{x**3}
				\tkzFct[domain=-3:3]{0.5*x + 1}
				\tkzDefPoints{
				1.1653730430624/1.5826865215312/P}
				\tkzDrawPoint(P)
				\tkzLabelPoints[above left](P)			
			\end{tikzpicture}
	\end{minipage}
	\begin{minipage}[!]{0.45\linewidth}
		\centering
		\caption{Exemplo 2}
		\label{exemplo_2}
			\begin{tikzpicture}
				\tkzInit[xmin = -3, xmax = 3, ymin = -1, ymax = 4 ]
				\tkzDrawX[label={$x$}, noticks]
				\tkzDrawY[label={$y$}, noticks]
				\tkzFct[domain=-4:3.5]{0.1*x**5 + x**2}
				\tkzFct[domain=-3:3]{0.33697408*x + 2.055869992}
				\tkzDefPoints{
				-1.64/1.503232501/P,
				1.4067607536057/2.5299119027264/Q}
				\tkzDrawPoint(P)
				\tkzLabelPoints[above](P)	
				\tkzDrawPoint(Q)	
				\tkzLabelPoints[above right](Q)			
			\end{tikzpicture}
	\end{minipage}
	\legend{\ABNTEXfontereduzida Fonte: Elaborada pelo autor, 2021.}
\end{figure}

Para uma definição que seja apropriada para as curvas em geral, primeiramente conceituaremos a inclinação da reta tangente no ponto. Na seção \ref{secao_1}, temos a Figura \ref{funcao_2} na qual há uma função $ y = f(x) $ e uma reta secante $s$. Consideremos o ponto $P_0$ fixo e o ponto $P_1$ móvel aproximando-se de $P_0$ pela sua direita. Conforme a variação de $\Delta x$ for reduzida, isto é, $x_1$ aproxima-se de $x_0$ o $\Delta x \rightarrow 0$ ($\Delta x$ tende a 0) e a inclinação de $s$ varia. Para uma maior generalização, considere o gráfico da função $y = f(x)$ da Figura \ref{funcao_3}, onde o ponto $Q(x_2, f(x_2))$ aproxima-se do ponto $P(x_1, f(x_1))$ fixo, tanto pela direita e pela esquerda de $P$. Sendo assim, o conceito de limite está presente, já que o ponto $Q$ está tendendo ao ponto $P$, fazendo com que a inclinação da reta secante $s$ se aproxime cada vez mais da inclinação da reta tangente $t$, ou seja, a reta tangente $t$ é a reta que passa por $P$ e tem inclinação $m$, sendo $m$ o limite da inclinação da reta secante $s$ (\ref{tangete_p_0}). 

Logo, temos a definição da reta tangente em um ponto $P$ em uma curva $y = f(x)$.

\begin{definicao}
	\label{def_tangente}
	\cite{leithold} Suponhamos que a função f seja contínua em $x_1$. A reta tangente ao gráfico de f no ponto $P(x_1, f(x_1))$ é
	
	(i) a reta por P tendo inclinação $m(x_1)$, dada por
	$$ m(x_1) = \lim_{\Delta x\rightarrow 0} \frac{f(x_1 + \Delta x) - f(x_1)}{\Delta x}$$
	
	se o limite existir;
	
	(ii) a reta x = $x_1$ se	
	$$ \lim_{\Delta x\rightarrow 0^{+}} \frac{f(x_1 + \Delta x) - f(x_1)}{\Delta x} \ for \  +\infty \  \text{ou} \ -\infty$$ \qquad \qquad \qquad e
	$$ \lim_{\Delta x\rightarrow 0^{-}} \frac{f(x_1 + \Delta x) - f(x_1)}{\Delta x}  \ for \  +\infty \  \text{ou} \ -\infty$$ 
	
\end{definicao}

\begin{figure}[h]
	\centering
	\caption{Gráfico de $y = f(x)$ com várias retas}
	\label{funcao_3}
	\begin{tikzpicture}
		\tkzInit[xmin = -4, xmax = 4, ymin = -1, ymax = 6]
		\tkzDrawX[label={$x$}, noticks]
		\tkzDrawY[label={$y$}, noticks]
		
		\tkzFct[color=green, domain=-4:4]{1/2**x}
		\tkzText(-1.5, 6){$y = f(x)$}
		\tkzFct[color = red, domain=-4:4]{-1.38629436*x + 0.61370638}
		\tkzText(-3.7, 5.25){$t$}
		
		\tkzDefPoints{
			-1/2/P,
			1/0.5/Q_2,
			2/0.25/Q_3,
			-2/4/Q_5,
			-2.48544/5.6011/Q_4}
		\tkzDrawPoints[color=black](P, Q_2, Q_3, Q_4, Q_5)
		\tkzText[color=black](-1.2, 1.75){$P$}
		
		\tkzText[color=black](-2.25, 5.6){$Q$}
		\tkzText[color=black](-1.6, 4){$Q$}
		\tkzText[color=black](1.3, 0.8){$Q$}
		\tkzText[color=black](2.3, 0.55){$Q$}
		%\tkzText[color=black](-1.6,2.3){$m$}
		%\tkzText(-0.4,1.7){$m$}
		
		\tkzDefPoints{
			-2/2/b,
			-2.5/2/a}
		\tkzDrawSegment[dashed](a,P)
		\tkzDrawSegment[dashed](a,Q_4)
		\tkzDrawSegment[dashed](b,Q_5)
		
		\tkzDefPoints{
			1/2/c,
			2/2/d}
		\tkzDrawSegment[dashed](P,d)
		\tkzDrawSegment[dashed](c,Q_2)
		\tkzDrawSegment[dashed](d,Q_3)
		
		
		%\tkzFct[domain=-4:4]{-x + 1}		
		\tkzFct[domain=-4:4]{-2*x}		
		\tkzFct[domain=-4:4]{(3.6*x + 0.63)/-1.49}		
		\tkzFct[domain=-4:4]{(-1.5*x + 2.5)/2}	
		\tkzFct[domain=-4:4]{(-1.75*x + 4.25)/3}
		
		
		
		
		%\tkzDrawPolygon(F, E, D)
		%angulo
		\draw(-1.5, 2) arc [radius = 0.5cm, start angle=180, end angle=125];
		%\draw(-0.5, 2) arc [radius = 0.5cm, start angle=0, end angle=-55];
		
		
				
	\end{tikzpicture}
	\legend{\ABNTEXfontereduzida Fonte: Elaborada pelo autor, 2021.}
\end{figure}

O limite apresentado na definição \ref{def_tangente} que diz respeito à inclinação de uma reta tangente $t$ em um ponto $P(x_1, f(x_1))$, pode ser usado para calcular qualquer ponto genérico. Esse limite é um dos mais importantes do Cálculo, denominado como derivada de uma função, no qual pode ser encontrado a partir da definição \ref{def_derivada}.

\begin{definicao}
	\label{def_derivada}
	\cite{leithold} A derivada de uma função f é a função denotada por f', tal que seu valor em qualquer número x do domínio de f seja dado por	
	$$ f'(x) = \lim_{\Delta x\rightarrow 0} \frac{f(x + \Delta x) - f(x)}{\Delta x}, $$
	se esse limite existir.
\end{definicao}

Com isso, percebemos que a derivada coincide com a inclinação da reta tangente $t$ no ponto $P$ e, assim, têm-se sua interpretação geométrica. 

Desta forma, agora é possível definir a taxa de variação instantânea. Note que, na taxa de variação média é possível calcular a variação média a partir de um intervalo dado, mas ao aplicar o limite a esta taxa de variação, o intervalo ficará cada vez menor fazendo com que o $\Delta x \rightarrow 0$, logo a variação em $x$ é ínfima, podendo-se calcular a variação no valor de $x$. A aplicação deste limite é denominado como taxa de variação instantânea, o que implica em outra interpretação da derivada, sendo ela a taxa de variação instantânea.

Existem algumas notações para a derivada, sendo as mais utilizadas: $f'(x)$, $y'$, $\dfrac{dy}{dx}$, $\dfrac{d}{dx}f(x)$ e $\dot{y}$.

O símbolo para derivada, $\dfrac{dy}{dx}$, foi introduzido por Leibniz e não pode ser interpretado como sendo o quociente entre $dy$ e $dx$. Já a notação, $f'(x)$, foi introduzida por Lagrange\footnote{De acordo com \citeonline{howard}, Joseph Louis Lagrange (1736-1813) foi um dos maiores matemáticos do século XVIII, que nasceu em Turim, Itália. Dentre os trabalhos e contribuições realizadas por Lagrange, além da sua notação $f'(x)$, temos o desenvolvimento da teoria das funções de variável real, um método de aproximação das raízes reais de uma equação por meio de frações contínuas, as equações de Lagrange e também grandes contribuições na Teoria dos Números.}. Ao calcular a derivada de uma função $y = f(x)$ com $x = x_1$, na notação de Lagrange e Leibniz temos $f'(x_1)$ e $\left. \dfrac{dy}{dx}\right| _{x = x_1}$, respectivamente. E Finalmente, temos a notação, $\dot{y}$, que foi introduzida por Newton.

\begin{exemplo}
	\label{exemplo_01}
	Calcule a derivada de $f(x) = x^{3}$. 
\end{exemplo}
\begin{solution}
	Pela definição \ref{def_derivada} temos	
	\begin{equation}
		\label{derivada}
		f'(x) = \lim_{\Delta x\rightarrow 0} \frac{f(x + \Delta x) - f(x)}{\Delta x}.
	\end{equation}
	
	A função dada pelo exemplo é 	
	\begin{equation}
		\label{func}
		f(x) = x^{3}, 
	\end{equation}

	logo temos que	
	\begin{equation}
		\label{x + delta}
		f(x + \Delta x) = (x + \Delta x)^{3}.
	\end{equation}
	
	Substituindo (\ref{func}) e (\ref{x + delta}) em (\ref{derivada}) têm-se
	$$f'(x) = \lim_{\Delta x\rightarrow 0} \frac{(x + \Delta x)^{3} - x^{3}}{\Delta x}$$	
	$$f'(x) = \lim_{\Delta x\rightarrow 0} \frac{x^{3} +3x^{2}\Delta x + (\Delta x)^{3} + 3x(\Delta x)^{2} - x^{3}}{\Delta x}$$	
	$$f'(x) = \lim_{\Delta x\rightarrow 0} \frac{\cancel{x^{3}} +3x^{2}\Delta x + (\Delta x)^{3} + 3x(\Delta x)^{2} - \cancel{x^{3}}}{\Delta x}$$
	
	Colocando $\Delta x$ em evidência	
	$$f'(x) = \lim_{\Delta x\rightarrow 0} \frac{\Delta x(3x^{2} + (\Delta x)^{2} + 3x\Delta x)}{\Delta x}$$	
	$$f'(x) = \lim_{\Delta x\rightarrow 0} \frac{\cancel{\Delta x}(3x^{2} + (\Delta x)^{2} + 3x\Delta x)}{\cancel{\Delta x}}$$
	$$f'(x) = \lim_{\Delta x\rightarrow 0} 3x^{2} + (\Delta x)^{2} + 3x\Delta x$$
	
	Aplicando o limite quando $\Delta x \rightarrow 0$, alguns termos irão zerar, logo têm-se
	$$f'(x) = 3x^{2} + (\cancel{\Delta x)^{2}} + \cancel{3x\Delta} x$$
	
	Portando, a derivada de $f(x)$ é	
	$$f'(x) = 3x^{2}$$	
			
\end{solution}

\section{Regras de derivação}
\label{secao_3}

Na seção \ref{secao_2} foi apresentado o exemplo \ref{exemplo_01}, no qual fica notório a trivialidade da derivação de (\ref{func}) e funções similares ou até mesmo funções mais simples com o uso da definição \ref{def_derivada}. Sendo assim, para calcular a derivada de uma função mais complexa, como por exemplo, $f(x) = x^{6} - x^{4} + x^{3} - x + 150$, seria muito trabalhoso usar a definição de derivada, apresentada na seção anterior, para determinarmos a derivada de $f(x)$. Sabe-se, entretanto, que existe outra maneira relativamente mais simples para calcular a derivada de uma função, que consiste em usar as regras de derivação que são obtidas e demonstradas a partir da definição de derivada. As regras de derivações foram definidas de acordo com \citeonline{leithold} e \citeonline{stewart}.

\begin{enumerate}
	%Derivada da constante
	\item \textbf{Derivada da Função Constante}
	\label{derivada_constante}
	
	Seja $c$ uma constante e $f(x) = c$ para todo valor de $x$, temos que	$$\frac{d}{dx}(c) = 0.$$
	Sendo assim, a derivada de uma constante é igual a zero.
	
	%Derivada da potência
	\item \textbf{Regra da Potência}
	\label{derivada_potencia}
	
	Seja uma função potência $f(x) = x^{n}$ derivável, tal que $n$ é uma constante, então
	$$\frac{d}{dx}(x^{n}) = nx^{n-1}.$$
	
	%derivada da multiplicação por constante
	\item \textbf{Regra da Multiplicação por Constante}
	\label{derivada_multiplicacao_constante}
	
	Seja $c$ uma constante, $f(x)$ uma função e $g(x)$ uma função definida como $g(x) = cf(x)$, têm-se que	
	$$\frac{d}{dx}g(x) = c \frac{d}{dx}f(x),$$
	se $\dfrac{d}{dx}f(x)$ existir.
	
	%regra da soma
	\item \textbf{Regra da Soma}
	\label{derivada_soma}
	
	Sejam $f(x)$ e $g(x)$ funções deriváveis, logo temos que
	$$\frac{d}{dx}[f(x) + g(x)] = \frac{d}{dx}f(x) + \frac{d}{dx}g(x).$$
	Sendo assim, a derivada da soma de duas funções é igual a soma das derivadas. Note que esta regra se estende para quaisquer números finitos de funções, onde a derivada da somas destas funções é igual a soma das derivadas das respectivas funções, desde que elas existam. 
	
	%regra do produto
	\item \textbf{Regra do Produto}
	
	Sejam $f(x)$ e $g(x)$ funções deriváveis e $h(x)$ uma função definida como sendo o produto de $f$ e $g$. Para alguém desavisado, seguindo a mesma linha de raciocínio da regra da soma(\ref{derivada_soma}), a derivada de $h(x)$ seria o produto das derivadas de $f(x)$ e $g(x)$, mas esta conjectura é errônea. Pode-se notar a falsidade desta afirmação ao usarmos um exemplo simples:
	
	Admita $f(x) = x^{3}$ e $g(x) = x^{4}$, logo $h(x) = x^{3}x^{4} = x^{7}$. Sendo assim, $f'(x) = 3x^{2}$ e $g'(x) = 4x^{3}$, de acordo com a regra da potência (\ref{derivada_potencia}). Efetuando a multiplicação de $f'(x)$ com $g'(x)$, têm-se	
	$$ f'(x)g'(x) = 3x^{2}4x^{3} = 12x^{5}, $$
	
	note que $h'(x) = 7x^{6}$ por (\ref{derivada_potencia}), então $f'(x)g'(x) \neq h'(x)$. O método correto para calcular a derivada do produto de duas funções é
	$$  \dfrac{d}{dx}[f(x)g(x)] = f(x)\dfrac{d}{dx}[g(x)] + g(x)\dfrac{d}{dx}[f(x)]. $$
	
	Este método foi descoberto por Leibniz e é conhecido como a Regra do Produto.
	
	%regra do quociente
	\item \textbf{Regra do Quociente}
	
	Assim como na Regra do Produto, o mesmo se verifica na Regra do Quociente, a derivada de um quociente não é o quociente das derivadas. Sejam $f(x)$ e $g(x)$ funções deriváveis, e $h(x)$ definida como o quociente de $f$ e $g$, isto é, $h(x) = \dfrac{f(x)}{g(x)}$. Utilizando um exemplo trivial para mostrar que $h'(x)$ não é o quociente das derivadas de $f$ e $g$, respectivamente:
	
	Considere $f(x) = x^{2}$ e $g(x) = x$, logo $f'(x) = 2x$ e $g'(x) = 1$ pela regra da potência (\ref{derivada_potencia}). Portanto, $$\dfrac{f'(x)}{g'(x)} = \dfrac{2x}{1} = 2x.$$
	
	Note que $h(x) = \dfrac{x^{2}}{x} = x$, logo $h'(x) = 1$ pela regra da potência, ou seja, $2x \neq 1$ o que implica em $\dfrac{f'(x)}{g'(x)} \neq h'(x).$ Desta forma, é possível observar que este não é o método correto para derivar $h(x)$, a forma correta dá-se pela regra do quociente	
	$$  \dfrac{d}{dx}\left[ \dfrac{f(x)}{g(x)} \right] = \dfrac{g(x)\dfrac{d}{dx}[f(x)] - f(x)\dfrac{d}{dx}[g(x)]}{[g(x)]^{2}}, $$ 
	
	desde que, $g(x) \neq 0$.
	
	\item \textbf{Derivada das Funções Trigonométricas}
	\label{derivada_trigonometricas}
	
	As funções trigonométricas são de suma importância em diversas aplicações, com elas é possível estudar fenômenos periódicos, propagação de ondas sonoras, etc. Essas funções, assim como as outras funções deriváveis, também possuem as suas respectivas derivadas, sendo elas:
	
	\begin{enumerate}
		\item $ \dfrac{d}{dx}[\text{sen}(x)] = \cos(x) $
		\label{derivada_seno}
		
		\item $ \dfrac{d}{dx}[\cos(x)] = -\text{sen}(x)$
		\label{derivada_cosseno}
		
		\item $ \dfrac{d}{dx}[\tan(x)] = \sec^{2}(x)$
		\label{derivada_tangente}
		
		\item $\dfrac{d}{dx}[\text{cossec}(x)] = -\text{cossec}(x)\text{cotg}(x)$
		\label{derivada_cossec}
		
		\item $\dfrac{d}{dx}[\text{sec}(x)] = \text{sec}(x)\tan(x)$
		\label{derivada_sec}
		
		\item $\dfrac{d}{dx}[\text{cotg}(x)] = -\text{cossec}^{2}(x)$
		\label{derivada_cotg}
	\end{enumerate}	
	
	%regra da cadeia
	\item \textbf{Regra da Cadeia}
	\label{derivada_cadeia}
	
	No caso em que temos $h(x)$, sendo $h(x)$ uma função composta definida como $h(x) = f \circ g$, não é possível calcular a derivada de $h(x)$ somente com as regras anteriores, para isso, é preciso utilizar um recurso conhecido como a Regra da Cadeia. Sendo assim, para que $h(x)$ seja derivável, $g$ tem que ser derivável no ponto $x$ e $f$ derivável no ponto $g(x)$. A Regra da Cadeia é um importante recurso do Cálculo Diferencial que possibilita derivar funções compostas, de tal forma que a derivada de uma função composta é o produto das derivadas de $f$ e $g$. Portanto, temos que
	$$ (f \circ g)'(x) = f'(g(x)) \cdot g'(x).$$	 
	
	Utilizando a notação de Leibniz para a derivada de uma função composta, considere $y = f(u)$ e $u = g(x)$, com isso têm-se
	
	$$ \dfrac{dy}{dx} = \dfrac{dy}{du}\dfrac{du}{dx}. $$
	
	\item \textbf{Derivada da Função Exponencial}
	\label{derivada_exponecial}
	
	Seja $f(x) = a^{x}$, com $a\in \mathbb{R}$, tal que $a > 0$ e $a \neq 1$, têm-se
	$$  \dfrac{d}{dx}[a^{x}] = a^{x}\ln a. $$
	
	Na  derivada da função exponencial existe um caso particular, onde dada a função $f(x) = e^{x}$, têm-se
	$$  \dfrac{d}{dx}[e^{x}] = e^{x}. $$
	
	\item \textbf{Derivada da Função Logarítmica}
	\label{derivada_logaritmica}
	
	Seja $f(x) = \log_b x$, com $x\in \mathbb{R}$, tal que $x > 0$, $b > 0$ e $b \neq 1$, temos que
	$$ \dfrac{d}{dx}[\log_b x] = \dfrac{1}{x}\log_b e.  $$
	
	Assim como na derivada da função exponencial, também existe um caso particular na derivada da função logarítmica, onde dada a função $f(x) = \ln x$, têm-se
	
	$$ \dfrac{d}{dx}[\ln x] = \dfrac{1}{x}.$$
	
\end{enumerate}

As regras de derivação presentes na seção \ref{secao_3} são todas demonstradas a partir da definição de derivada (\ref{def_derivada}). Vimos aqui, conceitos importantes que compõem o Cálculo Diferencial, sendo eles a taxa de variação média, taxa de variação instantânea, com suas respectivas interpretações geométricas e as regras de derivação. No próximo Capítulo abordaremos os assuntos relacionados às derivadas temporais, juntamente com algumas de suas aplicações na Física, que são o principal objeto de estudo deste trabalho.

\chapter{TAXA DE VARIAÇÃO TEMPORAL}
\label{capitulo_3}
\thispagestyle{empty}

O intuito deste Capítulo é mostrar algumas aplicações da taxa de variação temporal que estão presentes na Física. Para isso, primeiramente será abordado de modo simples a definição de velocidade e aceleração, diferenciando certos aspectos que englobam essas duas grandezas. Em seguida, introduziremos a definição da velocidade e aceleração instantâneas, que são taxas de variações temporais. Essa abordagem é de suma importância para chegarmos no objetivo deste trabalho, pois, essas taxas de variações são necessárias para que seja possível mostrar algumas aplicações referentes a este tema.

Para apresentarmos as definições presentes neste Capítulo, de forma mais rica, teremos como base os seguintes autores: \citeonline{fundamentos_fisica}, \citeonline{licoes_fisica}, \citeonline{fisica_conceitual}, \citeonline{fisica_contexto}, \citeonline{uma_abordagem}, \citeonline{moyses} e \citeonline{fisica_1}.

\section{Velocidade e aceleração}
\label{secao_velocidade_aceleracao}

A velocidade e aceleração são conceitos muito importantes na análise do movimento de quaisquer objetos, sejam eles partículas, corpos, automóveis, bolas, etc. De acordo com \citeonline{licoes_fisica}, o movimento é a mudança da posição com o tempo. Deste modo, podemos analisar o movimento através da velocidade e aceleração, onde é possível, por exemplo, verificar se um determinado objeto está em repouso, movimento acelerado ou movimento retrógrado. 

Para uma pessoa que não aprofundou-se na definição de repouso, é normal dizer que um objeto está em repouso quando ele está parado. Esta ideia em geral não está totalmente certa, pois, existem detalhes neste conceito que precisam ser analisados. Primeiramente, admita como exemplo a seguinte situação: uma pessoa dirigindo um automóvel a 50Km/h em uma avenida. A partir deste exemplo podemos fazer algumas análises. Observe que para uma pessoa de fora do automóvel em um referencial inercial, do seu ponto de vista o piloto está em movimento, ou seja, ele não está parado. Agora, admita a mesma situação com o diferencial que existe outra pessoa dentro do automóvel. Note que quando o passageiro observar o piloto do automóvel, ele irá verificar que o mesmo está parado, isto é, em repouso, seguindo a conjectura inicial no qual um objeto está em repouso quando ele está parado. Ora, mas o piloto está a 50Km/h e como foi analisado no primeiro caso, ele não pode estar em repouso. Sendo assim, como pode existir esta ambiguidade em relação ao repouso do piloto?

Este detalhe que precisa ser analisado ao definir o repouso, pois o repouso nada mais é do que o movimento compartilhado, no qual o mesmo não é percebido. Deste modo, observe que em determinado referencial a pessoa conduzindo o automóvel está em movimento e em outro referencial ela está em repouso. Portanto, o repouso depende de qual referencial está sendo observada determinada situação. 

O movimento acelerado ocorre quando há uma mudança na posição com o tempo, só que essa mudança na posição é cada vez maior no decorrer do tempo, ou seja, o objeto está acelerando. Exemplo: uma pessoa em cima de um prédio deixa uma bola cair, inicialmente a velocidade desta bola é igual a zero, mas no decorrer do tempo a sua velocidade aumenta até colidir. 

O movimento retrógrado é o inverso do acelerado, onde a mudança na posição é cada vez menor no decorrer do tempo, um exemplo clássico é um automóvel em movimento que tem os seus freios acionados.

Atualmente quando fala-se sobre velocidade ou aceleração, são conceitos que dificilmente são desconhecidos pelo público em geral, mas não necessariamente a sua definição formal, mas sim a sua ideia intuitiva. Fato que era desconhecido até mesmo pelos antigos. 

\begin{citacao}
	No século XVII, é claro que ninguém sabia nem se preocupava com coisa alguma sobre beisebol. Mas entender como a posição, a velocidade e a trajetória de uma bola arremessada estão num estado constante de variação é a base para entender a física de todos os corpos em movimento. Nesse sentido, o aceleração variável, por exemplo, teria sido um conceito difícil de apreender para um matemático grego antigo, pois é a medida da variação da velocidade no tempo, enquanto a velocidade é uma medida da variação da posição com o tempo \cite[p. 23]{bardi}.
\end{citacao}

Com isso, fica notório que esses conceitos conhecidos pela maioria das pessoas, antigamente era desconhecido ou algo complexo de entender. Antes das definições formais de velocidade e aceleração serem abordadas, será abordado primeiramente sobre \textit{rapidez}\footnote{Também conhecida como velocidade escalar.}. Antes de Galileu definir a rapidez de um objeto, os antigos utilizavam os termos "lento"$ \ $ou "rápido"$ \ $para descrever o movimento de um objeto. Segundo \citeonline{fisica_conceitual}, a rapidez foi definida como sendo a distância percorrida por unidade de tempo. 
$$\mathrm{rapidez = \dfrac{dist \hat{a}ncia}{tempo}}$$

A partir desta definição é possível saber quão rápido é o movimento de um objeto. Quando trata-se da rapidez, temos a rapidez média e a rapidez instantânea. A rapidez média é obtida através da distância total percorrida por intervalo de tempo, ou seja, ela é definida como sendo
$$\mathrm{rapidez \ m \acute{e}dia= \dfrac{dist \hat{a}ncia \ total \ percorrida}{intervalo \ de \ tempo}.}$$

Considere como exemplo, uma pessoa que precisa percorrer uma distância de 800 metros a pé para chegar em seu trabalho no intervalo de tempo de 16 minutos, com isso, a rapidez média que ela precisa movimentar-se é
$$ \mathrm{rapidez \ m \acute{e}dia = \dfrac{800}{16} = 50 \ m/min.}$$

Logo, a rapidez média dessa pessoa precisa ser igual a 50 m/min para percorrer a distância de 800 metros em 16 minutos. Já a rapidez instantânea é a rapidez exata no instante de tempo $t$.

Novamente, uma pessoa que desconhece esses conceitos relacionados à rapidez, ficará confusa, pois as conjecturas anteriores são conhecidas no cotidiano como velocidade. Ora, realmente a definição de rapidez assemelha-se com a da velocidade. Entretanto, há uma diferença entre rapidez e velocidade, no qual a velocidade é uma grandeza vetorial, possuindo módulo, sentido e direção. Contudo, a rapidez trata-se de uma grandeza escalar, possuindo apenas o módulo. Logo, o que pode-se concluir a grosso modo, é que a velocidade é uma rapidez com orientação. 

De acordo com \citeonline{fundamentos_fisica}, a velocidade média de um determinado objeto pode ser obtida pelo quociente da variação do espaço ($\Delta s$) em relação à variação do tempo ($\Delta t$).
\begin{equation}
	\label{velocidade_media}
	\text{velocidade média} = \dfrac{\Delta s}{\Delta t} = \dfrac{s_1 - s_0}{t_1 - t_0}
\end{equation}

Na velocidade, existe a velocidade constante e a velocidade variável, no qual indicam informações importantes a respeito do movimento de um objeto. Primeiramente, no primeiro caso, a velocidade constante indica que não há mudança na rapidez e nem na orientação, fazendo com que o objeto permaneça com a mesma velocidade durante o deslocamento. No caso da velocidade variável, como o próprio nome induz, há uma variação na velocidade que é denominada como \textit{aceleração}. De acordo com \citeonline{fisica_conceitual}, a aceleração é a variação da velocidade por intervalo de tempo, isto é, para que haja uma variação na velocidade é necessário que tenha uma variação na rapidez ou na orientação, também sendo possível em ambas ao mesmo tempo, acarretando assim em uma mudança de velocidade. Note que, na velocidade constante, não há variação da velocidade, ou seja, a aceleração é igual a zero. Assim como a velocidade, a aceleração também é uma grandeza vetorial.
$$\mathrm{acelera \text{ç} \tilde{a}o = \dfrac{varia \text{çã}o \ da \ velocidade}{intervalo \ de \ tempo}}$$

Para contextualizar essas conjecturas, admita como exemplo, um ônibus que irá sair do terminal. Inicialmente, o ônibus está em repouso, concluindo-se assim que ele tem velocidade constante igual a zero. Entretanto, quando o motorista do ônibus começa a entrar em sua respectiva rota, há uma mudança na velocidade, que anteriormente era constante e possuía aceleração igual a zero. Ao começar a sair do terminal para entrar em rota, o ônibus será deslocado e terá uma aceleração $x$. Doravante, quando o ônibus está em rota ele possui uma velocidade $y$, que em determinado momento pode ser uma velocidade constante ou uma velocidade variável. A velocidade variável fica óbvia em determinados momentos, sendo eles, a parada do ônibus em algum ponto ou semáforo, onde os freios do mesmo são acionados, com o intuito de que haja uma mudança de velocidade e o ônibus pare. Neste caso, a aceleração terá uma orientação contrária a do ônibus, reduzindo-se assim a sua velocidade.

\section{Velocidade e aceleração instantâneas}
\label{secao_3.2}

Primeiramente, consideremos o tipo de movimento mais simples que existe na Física, o movimento retilíneo uniforme (MRU). Inicialmente, admitiremos a seguinte situação: Um observador localizado em uma calçada, analisa o movimento de um veículo que começa a entrar no quarteirão até o final do mesmo. A partir de sua análise, o observador conclui que o veículo estava em MRU e com isso construiu uma tabela (Tabela \ref{tabela_exemplo}) com a distância percorrida pelo veículo em determinados instantes de tempo. Para isso, ele considerou a distância e o tempo sendo zero a partir do momento em que o veículo entrou no quarteirão.

\begin{table}[!h]
	\centering
	\caption{Distância percorrida em relação ao tempo}
	\label{tabela_exemplo}
	\begin{tabular}{c|c}
		\hline
		Tempo & Distância \\ \hline
		0s & 0m \\ \hline
		1s & 20m \\ \hline
		2s & 40m \\ \hline
		3s & 60m \\ \hline
		4s & 80m \\ \hline
		5s & 100m \\ \hline	
	\end{tabular}
	\legend{\ABNTEXfontereduzida Fonte: Elaborada pelo autor, 2021.}
\end{table}

Sendo assim, por tratar-se de um movimento retilíneo uniforme, é possível analisar de acordo com a Tabela \ref{tabela_exemplo}, que em cada instante $t$ a posição varia em 20 metros, implicando assim em uma velocidade igual a 20m/s.

Utilizando um sistema de coordenadas (Figura \ref{sistema}), a fim de ilustrar a situação, considere o eixo $Om$, no qual representa o logradouro percorrido pelo veículo, contendo as respectivas distâncias percorridas em cada instante $t$. Deste modo, a origem $O$ deste sistema de coordenadas, nesta situação, trata-se do início do quarteirão.  

\begin{figure}[!h]
	\centering
	\caption{Sistema de coordenadas do exemplo}
		\begin{tikzpicture}
			\tkzInit[xmin= -1, xmax= 5.5, ymin= -0.5, ymax= 1]
			\tkzDrawY[label={$$}, noticks]
			\tkzDrawX[label={$m$}]
			\tkzText(-0.2, -0.2){$O$}
			\tkzLabelPoint[below](1, 0){20}
			\tkzLabelPoint[below](2, 0){40}
			\tkzLabelPoint[below](3, 0){60}
			\tkzLabelPoint[below](4, 0){80}
			\tkzLabelPoint[below](5, 0){100}
			
		\end{tikzpicture}
	\label{sistema}
	\legend{\ABNTEXfontereduzida Fonte: Elaborada pelo autor, 2021.}
\end{figure}

Considerando que o veículo está indo no sentido positivo do eixo, isto é, da esquerda para a direita, a sua velocidade média é 20m/s. Entretanto, se fosse considerado o contrário, da direita para esquerda, o veículo estaria se movendo no sentido negativo do eixo. A partir disso, temos que a velocidade média do veículo, de acordo com \ref{velocidade_media}, é igual a -20m/s, pois neste caso a posição inicial e a posição final são iguais a 100 e a 0 metros, respectivamente. Logo
$$\dfrac{\Delta s}{\Delta t} = \dfrac{0 - 100}{5 - 0} = -20\text{m/s}.$$ 

Neste caso, um valor negativo ou positivo depende do sentido definido para o eixo. Já que a velocidade é uma grandeza vetorial, o sinal indica o sentido no qual o veículo está sendo conduzindo. Entretanto, o quão rápido ele está indo trata-se de uma grandeza escalar, isto é, a rapidez do veículo. Sendo assim, ela é a mesma para ambos os casos, ou seja, 20m/s. Esta análise é interessante, já que mostra de modo concreto a diferença entre rapidez e velocidade.

Considerando o sentido positivo do eixo e os dados apresentados na Tabela \ref{tabela_exemplo}, é possível encontrar a função que relaciona a posição com o tempo, a partir da função horária da posição do MRU, que é dada por
\begin{equation}
	\label{funcao_horaria}
	S(t) = s_0 + vt,
\end{equation}
onde $s_0$ é a posição inicial, $v$ é a velocidade, $t$ é o instante de tempo e $S(t)$ é a posição em determinado instante $t$. Como está sendo considerado que o veículo está indo no sentido positivo do eixo, têm-se que $s_0 = 0$ e $v = 20$, logo substituindo esses dados em (\ref{funcao_horaria}), temos
\begin{equation}
	\label{funcao_horaria_exemplo}
	S(t) = 20t.
\end{equation}
A partir da função \ref{funcao_horaria_exemplo}, é possível determinar a posição do veículo em qualquer instante de tempo, no qual o veículo necessita para fazer o seu trajeto até o final do quarteirão. Na figura \ref{grafico_st}, está presente o representação gráfica de $S(t)$ no intervalo $0 \leq t \leq 5$.

\begin{figure}[!h]
	\centering
	\caption{Gráfico de $S(t)$}
	\begin{tikzpicture}
		\tkzInit[xmin= -0.9, xmax= 4.5, ymin= -0.6, ymax= 6]
		\tkzDrawY[label={$s$}, noticks]
		\tkzDrawX[label={$t$}, noticks]
		\tkzDefPoints{
			0/0/t_0,
			0.8/1.2/t_1,
			1.6/2.4/t_2,
			2.4/3.6/t_3,
			3.2/4.8/t_4,
			4/6/t_5}
		\tkzDrawPoints(t_0, t_1, t_2, t_3, t_4, t_5)
		\tkzDrawSegment(t_0, t_5)
		\tkzDefPoints{
			0/0/tt_0,
			0.8/0/tt_1,
			1.6/0/tt_2,
			2.4/0/tt_3,
			3.2/0/tt_4,
			4/0/tt_5,
			0/1.2/ttt_1,
			0/2.4/ttt_2,
			0/3.6/ttt_3,
			0/4.8/ttt_4,
			0/6/ttt_5}
		\tkzLabelPoint[below right](tt_0){0}
		\tkzLabelPoint[below](tt_1){1}
		\tkzLabelPoint[below](tt_2){2}
		\tkzLabelPoint[below](tt_3){3}
		\tkzLabelPoint[below](tt_4){4}
		\tkzLabelPoint[below](tt_5){5}
		\tkzLabelPoint[left](ttt_1){20}
		\tkzLabelPoint[left](ttt_2){40}
		\tkzLabelPoint[left](ttt_3){60}
		\tkzLabelPoint[left](ttt_4){80}
		\tkzLabelPoint[left](ttt_5){100}
		
		\tkzDefPoints{
			0/1.2/y_1,
			0/2.4/y_2,
			0/3.6/y_3,
			0/4.8/y_4,
			0/6/y_5}
		
		\tkzDrawSegment[dashed](t_1, y_1)
		\tkzDrawSegment[dashed](t_2, y_2)
		\tkzDrawSegment[dashed](t_3, y_3)
		\tkzDrawSegment[dashed](t_4, y_4)
		\tkzDrawSegment[dashed](t_5, y_5)
		\tkzDrawSegment[dashed](tt_1, t_1)
		\tkzDrawSegment[dashed](tt_2, t_2)
		\tkzDrawSegment[dashed](tt_3, t_3)
		\tkzDrawSegment[dashed](tt_4, t_4)
		\tkzDrawSegment[dashed](tt_5, t_5)

	\end{tikzpicture}
	\label{grafico_st}
	\legend{\ABNTEXfontereduzida Fonte: Elaborada pelo autor, 2021.}
\end{figure}

Note que, como o movimento do veículo é um MRU, o gráfico de $S(t)$ é uma reta. Agora, consideremos as conjecturas apresentadas no Capítulo \ref{capitulo_2}, primeiramente com foco na seção \ref{secao_1}. Nesta seção, é introduzida a taxa de variação média (\ref{taxa_med}) e, ao anexarmos este conceito à situação anterior, podemos calcular a taxa de variação média de $S(t)$ em qualquer intervalo de tempo dado, ou seja, podemos calcular a velocidade média. Com isto em mente, admita o intervalo $2 \leq t \leq 4 \Rightarrow S(2) = 40 \ \text{e} \ S(4) = 80.$ Logo, substituindo em (\ref{taxa_med}), têm-se exatamente a velocidade média do veículo
$$\dfrac{S(t_1) - S(t_0)}{t_1 - t_0} = \dfrac{80 - 40}{4 - 2} = \dfrac{40}{2} = 20\text{m/s}.$$

Em determinadas situações, somente a velocidade média é necessária para analisar o movimento de um objeto. Contudo, existem situações onde é preciso saber a velocidade em determinado instante específico, que não é possível determinar somente com o uso da velocidade média. Logo, para que seja possível obter a velocidade em um exato instante, é preciso primeiramente defini-la, onde tal velocidade é denominada como velocidade instantânea.  

A velocidade instantânea é definida a partir do mesmo processo abordado na seção \ref{secao_2}, no qual a taxa de variação instantânea (\ref{def_derivada}) é obtida ao aplicar o limite na taxa de variação média. Portanto, ao aplicar o limite na velocidade média com $t_1 \rightarrow t_0$, isto é, $\Delta t \rightarrow 0$, o intervalo de tempo será cada vez menor, acarretando em uma variação na velocidade, tendo em vista que, esta variação aproxima-se cada vez mais da velocidade no instante $t$. Assim, temos a taxa de variação temporal\footnote{"TEMPORAL. Do lat. \textit{temporale}, que significa temporário, relativo a tempo, das fontes da cabeça."\cite[p. 722]{dic}} da posição, ou seja, a velocidade instantânea, que é definida como sendo o limite da velocidade média, segundo \citeonline{fisica_1}.
\begin{equation}
	\text{velocidade instantânea} = \lim_{\Delta t\rightarrow 0} \dfrac{\Delta s}{\Delta t}
\end{equation}

Considere o gráfico da Figura \ref{grafi}. Graficamente, a velocidade média é a inclinação da reta secante em determinados pontos. Quando aplicamos o limite na velocidade média, a variação no tempo irá tender a zero, e as inclinações das retas secantes irão aproximar-se cada vez mais da velocidade no instante $t$. Sendo assim, a velocidade instantânea é exatamente a inclinação da reta tangente ao ponto $P$.

\begin{figure}[h]
	
	\centering
	\caption{Interpretação geométrica}
	\label{funcao_generica}
	\begin{tikzpicture}
		
		\tkzInit[xmin = -1, xmax = 4, ymin = -0.5, ymax = 4.7]
		\tkzDrawX[label={$t$}, noticks]
		\tkzDrawY[label={$S(t)$}, noticks]
		
		\tkzFct[color=green, domain=0:4]{0.7*x**2}
		\tkzFct[color = red, domain=-1:4]{1.4*x - 0.7}
		
		\tkzDefPoints{
			1/0.7/P,
			2/2.8/Q_1,
			2.5/4.375/Q_2,
			2/0.7/a,
			2.5/0.7/b}	
		
		\tkzDrawPoints(P, Q_1, Q_2)
		\tkzLabelPoints[left](P, Q_1, Q_2)	
		\tkzDrawSegment[dashed](P, b)
		\tkzDrawSegment[dashed](Q_1, a)
		\tkzDrawSegment[dashed](Q_2, b)
		
		\tkzFct[domain=-1:4]{2.1*x -1.4}
		\tkzFct[domain=-1:4]{(3.68*x -2.63)/1.5}
		
		\tkzText[color=black](1, -0.3){$t$}
		\tkzText[color=black](2, -0.3){$t_1$}
		\tkzText[color=black](2.5, -0.3){$t_2$}
			
		
		%\tkzDrawPolygon(F, E, D)
		%angulo
		\draw(1.6, 0.7) arc [radius = 0.5cm, start angle=0, end angle=64];
		
	\end{tikzpicture}
	\label{grafi}
	\legend{\ABNTEXfontereduzida Fonte: Elaborada pelo autor, 2021.}
\end{figure}

De modo simples, a velocidade instantânea de um objeto é encontrada ao calcular a derivada de uma função $S(t)$ no instante $t$. Sendo assim, retornando ao exemplo anterior, temos a função horária da posição (\ref{funcao_horaria_exemplo}). Portanto, a partir das regras de derivação da seção \ref{secao_3}, temos que
\begin{equation}
	S'(t) = 20.
\end{equation}

Note que $S'(t)$ é uma função constante. Sendo assim, graficamente temos o gráfico da velocidade \textit{versus} tempo, no qual o mesmo é uma reta paralela ao eixo $t$ no intervalo [0, 5], como pode ser observado na Figura \ref{grafico_constante}. Assim, para qualquer instante $t$ do intervalo de tempo, a velocidade instantânea do veículo é igual a 20m/s, o que é evidente, visto que o movimento do veículo é um MRU.

\begin{figure}[!h]
	\centering
	\caption{Gráfico de $S'(t)$}
	\begin{tikzpicture}
		\tkzInit[xmin= -0.9, xmax= 5.5, ymin= -0.9, ymax= 5.5]
		\tkzDrawY[label={$v$}, noticks]
		\tkzDrawX[label={$t$}]
		
		\tkzFct[domain=0:5]{y = 5}
		
		\tkzLabelPoint[left](0, 5){20}
		
		\tkzDefPoints{
			0/0/tt_0,
			1/0/tt_1,
			2/0/tt_2,
			3/0/tt_3,
			4/0/tt_4,
			5/0/tt_5}
		\tkzLabelPoint[below right](tt_0){0}
		\tkzLabelPoint[below](tt_1){1}
		\tkzLabelPoint[below](tt_2){2}
		\tkzLabelPoint[below](tt_3){3}
		\tkzLabelPoint[below](tt_4){4}
		\tkzLabelPoint[below](tt_5){5}
		
		\tkzDefPoints{
			0/5/t_0,
			1/5/t_1,
			2/5/t_2,
			3/5/t_3,
			4/5/t_4,
			5/5/t_5}
		
		\tkzDrawPoints(t_0, t_1, t_2, t_3, t_4, t_5)
		
		\tkzDrawSegment[dashed](tt_1, t_1)
		\tkzDrawSegment[dashed](tt_2, t_2)
		\tkzDrawSegment[dashed](tt_3, t_3)
		\tkzDrawSegment[dashed](tt_4, t_4)
		\tkzDrawSegment[dashed](tt_5, t_5)
		
	\end{tikzpicture}
	\label{grafico_constante}
	\legend{\ABNTEXfontereduzida Fonte: Elaborada pelo autor, 2021.}
\end{figure}

A fim de continuar explorando as conjecturas anteriores, admita o seguinte exemplo: Uma pessoa no topo de uma torre deixa uma pedra cair verticalmente, enquanto um observador no solo cronometra o tempo necessário para que a mesma caia em um rio. Para isso, considere que a pedra caiu no rio após 4s, estando em queda livre\footnote{Quando um corpo está sujeito apenas à atração gravitacional, diz-se que ele está em queda livre."\cite[p. 85]{fundamentos_fisica}}.

Nesta situação, temos um movimento uniformemente variado, com isso, segue que função horária da posição é dada por
\begin{equation}
	\label{muv}
	S(t) = s_0 + v_0t + \dfrac{1}{2}gt^{2},
\end{equation}
onde neste exemplo, $S(t)$ é a quantidade de metros que a pedra caiu até colidir com o rio, $s_0$ é a posição inicial, $v_0$ é a velocidade inicial, $t$ é o tempo e $g$ é a aceleração da gravidade. Dito isso, consideremos $s_0 = 0, \ v_0 = 0$ e $g = 9,8\text{m/s}^{2}$. Substituindo estes dados em \ref{muv}, temos uma função do segundo grau, onde a quantidade de metros que a pedra caiu depende do tempo.
\begin{equation}
	\label{4,9_st}
	S(t) = 4,9t^{2}
\end{equation}

Deste modo, é possível encontrar quantos metros a pedra caiu no respectivo instante $t$, sendo preciso, para isso, apenas substituir o instante $t$ do intervalo de tempo em $S(t)$. Sendo assim, para descobrir quantos metros a pedra percorreu até cair dentro do rio, basta calcular $S(4)$, já que 4s foi o tempo necessário para tal ação, logo
\begin{center}
	$S(4) = 4,9(4)^{2}$
	
	$S(4) = 4,9 \cdot 16$
	
	$S(4) = 78,4 \ \text{metros}.$
\end{center}

Tendo em vista que a pedra possuí uma certa velocidade, é possível calcular a sua velocidade média, assim como a sua velocidade no exato instante. Deixando a velocidade média de lado, focamos na velocidade instantânea. Como foi visto anteriormente, ela trata-se da taxa de variação temporal da posição, ou seja, a taxa de variação da posição em relação ao tempo. Portanto, como temos a função que relaciona a posição da pedra em relação ao tempo (\ref{4,9_st}), apenas é preciso derivar a mesma.

Deste modo, ao derivar a função $S(t)$ obtêm-se a função da velocidade em relação ao tempo, isto é, $S'(t) = 9,8t$. Portanto, para qualquer valor de $t$ é possível adquirir a respectiva velocidade instantânea da pedra, desde que este valor de $t$ esteja no intervalo de tempo necessário para que a pedra caia no rio.

A Figura \ref{9,8t} contém a representação gráfica de $S'(t)$ no intervalo $0 \leq t \leq 4$. Observe que, $S'(t)$ é uma função afim. Portanto, o seu gráfico é uma reta que possui coeficiente angular igual a 9,8. Sendo assim, comparando os gráficos das Figuras \ref{grafico_constante} e \ref{9,8t}, percebe-se que no primeiro exemplo, para qualquer valor de $t$ a velocidade instantânea é sempre a mesma, e no segundo exemplo, a velocidade instantânea varia. Ora, isso acontece devido a aceleração, que foi introduzida na seção anterior.

\begin{figure}[!h]
	\centering
	\caption{Gráfico de $S'(t) = 9,8t$}
	\begin{tikzpicture}
		\tkzInit[xmin= -0.1, xmax= 4.9, ymin= -0.1, ymax= 4.9]
		\tkzDrawX[label={$t$}, noticks]
		\tkzDrawY[label={$v$}, noticks]
		\tkzFct[domain=0:4]{9.8*x}
		
		%\draw[step=0.5cm, black, thick]
		%--(0,1) coordinate(a)
		%--(0,0) coordinate(b)
		%--(1,0) coordinate(c)
		%pic["$\alpha$", draw=red, -, angle eccentricity=1.5, angle radius=0.45cm]{angle=a--b--c};
		\tkzFct[domain=0:2]{y=et}
		
	\end{tikzpicture}	
	\legend{\ABNTEXfontereduzida Fonte: Elaborada pelo autor, 2021.}
	\label{9,8t}
\end{figure}

Com os exemplos anteriores fica fácil perceber o modo como devemos encontrar a velocidade instantânea de um objeto a partir da sua função da posição em relação ao tempo. Contudo, quando trata-se de encontrar a aceleração da pedra em um determinado instante $t$, como é possível obter este resultado? 

Primeiramente é preciso definir tal aceleração. Na seção \ref{secao_velocidade_aceleracao} foi definida a aceleração como sendo a variação da velocidade por um instante de tempo. Entretanto, esta definição é para obter uma aceleração média de um objeto. Portanto, não é possível utilizar este método para encontrar a aceleração em um exato momento, mas de modo análogo à velocidade instantânea é possível encontrar tal resultado. Sendo assim, de acordo com \citeonline{fisica_1}, ela é encontrada a partir do limite da aceleração média ou simplesmente a derivada de segunda ordem\footnote{É a derivada da derivada, ou seja, $\dfrac{d}{dt}\left( \dfrac{ds}{dt} \right) = \dfrac{d^{2}s}{dt^{2}}$} da função da posição em relação ao tempo, onde a mesma é denominada como aceleração instantânea, ou seja, a taxa de variação temporal da velocidade.
\begin{equation}
	\text{aceleração instantânea} = \lim_{\Delta t\rightarrow 0} \frac{\Delta v}{\Delta t} = \dfrac{d^2s}{dt^2}
\end{equation}

Graficamente, a interpretação da aceleração instantânea é análoga à interpretação da velocidade instantânea. O diferencial é que, quando tratamos da velocidade instantânea, temos um gráfico da posição em relação ao tempo e, a partir disso, encontramos a inclinação da reta tangente em um ponto $P$, sendo esta inclinação a velocidade. Deste modo, quando tratamos da aceleração instantânea, temos um gráfico da velocidade em relação ao tempo. Sendo assim, a aceleração instantânea é a inclinação da reta tangente a um ponto $P$ no gráfico da função horária da velocidade.

Dito isso, agora é possível encontrar a aceleração instantânea da pedra que foi introduzida em um exemplo anteriormente. Para isso, simplesmente é preciso derivar $S'(t)$ obtendo-se assim a derivada de segunda ordem de $S(t)$. 
\begin{equation}
	\label{derivada_segunda}
	S''(t) = 9,8
\end{equation}

Note que derivando $S'(t)$ temos uma função constante (\ref{derivada_segunda}). Sendo assim, a aceleração instantânea para qualquer valor de $t$ no intervalo dado é de 9,8m/s$^2$, no qual tal aceleração é exatamente a da gravidade. Ora, o que faz sentido, pois inicialmente a pedra estava em repouso e logo em seguida foi deixada cair verticalmente em queda livre e a única força que está atuando sobre a pedra é a da gravidade.

\section{Aplicações}

Na seção anterior falamos sobre a velocidade e aceleração instantânea que são obtidas através da taxa de variação temporal. Com isso, nesta seção serão introduzidas algumas aplicações interessantes que são encontradas na Física.

\subsection{Cinemática}

Segundo \citeonline{fisica_contexto}, a cinemática é o ramo da mecânica que estuda o movimento dos corpos sem levar em consideração as causas para tornarem estes movimentos possíveis. Em situações apresentadas anteriormente, podemos notar a presença de movimentos pertencentes à cinemática, tento a título de exemplo, a primeira situação apresentada na seção \ref{secao_3.2} no qual há um veículo em movimento retilíneo uniforme.

Ainda na seção \ref{secao_3.2}, podemos notar a aplicação da taxa de variação temporal em determinadas situações, onde são encontradas as velocidades e acelerações instantâneas. Contudo, é possível aplicar estes conceitos de maneira mais ampla, de tal forma que os mesmos possam ser aplicados em quaisquer tipos de movimentos da cinemática.

Inicialmente, podemos recorrer a todo movimento classificado como sendo um MRU. Novamente, este tipo de movimento já foi introduzido anteriormente, mas a fim de mostrar que é possível aplicar os conceitos de velocidade e aceleração instantâneas em quaisquer situações, onde está presente um movimento caracterizado como sendo um MRU e não apenas em situações específicas do mesmo, ele será abordado novamente.

É denominado como sendo movimento retilíneo uniforme, ou simplesmente MRU, todo movimento realizado por um corpo em velocidade constante, no qual o mesmo percorre uma trajetória em linha reta. Por isso, ele recebe este nome, onde "retilíneo" refere-se à trajetória em linha reta e "uniforme" é referente à velocidade que é sempre constante. A partir de simples informações referentes ao movimento de algum corpo, é possível determinar a função horária da posição no MRU (\ref{funcao_horaria}).

Agora, aplicando a taxa de variação temporal nesta função (\ref{funcao_horaria}):
$$S(t) = s_0 + vt \\$$
$$\dfrac{dS}{dt} = \dfrac{d}{dt}[s_0 + vt]\\$$

Note que, $s_0$ e $v$ são constantes e utilizando as regras de derivação temos:
$$\dfrac{dS}{dt} = \dfrac{d}{dt}[s_0] + \dfrac{d}{dt}[vt]$$
$$\dfrac{dS}{dt} = 0 + v\dfrac{d}{dt}[t]$$
$$\dfrac{dS}{dt} = v$$

Portanto, a velocidade instantânea de um MRU é simplesmente o coeficiente angular da sua função horária da posição. Em relação a sua aceleração instantânea:
$$\dfrac{d^{2}S}{dt^{2}} = 0$$

De fato, a aceleração instantânea ser sempre zero é uma das características pertencentes ao MRU, já que a sua velocidade em todo o momento é constante.

Partindo para outro movimento da cinemática, denominado como movimento retilíneo uniformemente variado, ou simplesmente MRUV, é um movimento similar ao introduzido anteriormente com uma simples diferença que influencia em todo o movimento. Sendo assim, como no MRU, "retilíneo" refere-se à trajetória do corpo que é ao longo de uma reta e o diferencial no caso do MRUV é refente ao "uniformemente variado". Desse modo, existe uma variação que ocorre na velocidade devido a aceleração ser diferente de zero, onde a velocidade muda constantemente de modo uniforme em relação ao tempo. 

A função horária da posição no MRUV (\ref{muv}) já foi introduzida em uma situação anteriormente. Contudo, no exemplo no qual ela foi utilizada, a aceleração presente era a da gravidade por tratar-se de um uma trajetória em linha reta na vertical. Sendo assim, não necessariamente a aceleração da gravidade sempre será considerada. Para isso, considere $a$ como sendo a generalização da aceleração:
\begin{equation}
	\label{muv_2}
	S(t) = s_0 + v_0t + \dfrac{1}{2}at^{2}
\end{equation}

Portando, de modo análogo, para determinarmos a velocidade e aceleração instantânea de um MRUV, será aplicada a taxa de variação temporal nesta função (\ref{muv_2}):
$$\dfrac{dS}{dt} = \dfrac{d}{dt}\left[s_0 + v_0t + \dfrac{at^{2}}{2}\right]$$

Neste caso, $s_0$, $v_0$ e $a$ são constantes. Agora, derivando de acordo com as regras de derivação, temos:
$$\dfrac{dS}{dt} = \dfrac{d}{dt}[s_0] + \dfrac{d}{dt}[v_0t] + \dfrac{a}{2} \dfrac{d}{dt}\left[t^{2}\right]$$
$$\dfrac{dS}{dt} = 0 + v_0 + \dfrac{a}{2}2t$$
$$\dfrac{dS}{dt} = v_0 + at$$

Deste modo, temos a função horária da velocidade no MRUV que determina a velocidade instantânea de um corpo em determinante instante $t$. Tendo em vista que o MRUV é um movimento com aceleração constante, fica óbvio a partir das informações iniciais que a aceleração instantânea em qualquer instante é sempre igual a $a$. Contudo, partindo do pressuposto que a princípio temos apenas a função horária da velocidade no MRUV e precisamos determinar a aceleração instantânea de algum corpo, é suficiente calcular a taxa de variação temporal desta função.

Então, considerando a função horária da velocidade encontrada anteriormente, a mesma será derivada utilizando as regras de derivação:
$$\dfrac{d^{2}S}{dt^{2}} = \dfrac{d}{dt}[v_0 + at]$$
$$\dfrac{d^{2}S}{dt^{2}} = \dfrac{d}{dt}[v_0] + \dfrac{d}{dt}[at]$$
$$\dfrac{d^{2}S}{dt^{2}} = 0 + a\dfrac{d}{dt}[t]$$
$$\dfrac{d^{2}S}{dt^{2}} = a$$

Por fim, obtemos a aceleração instantânea partindo da função horária da velocidade.

\subsection{Segunda Lei de Newton}
%Momento linear

A 2ª Lei de Newton, também conhecida como o Princípio Fundamental da Dinâmica, é dada pela definição abaixo.

\begin{definicao}
	\label{definicao_2_lei}
	\cite{uma_abordagem}
	Um corpo de massa $m$, sujeito a forças $\vec{F_1}, \ \vec{F_2}, \ \vec{F_3}, \cdots$ sofrerá uma  aceleração $\vec{a}$ dada por	
	$$\vec{a} = \dfrac{\vec{F_{res}}}{m}$$	
	onde a força resultante $\vec{F_{res}} = \vec{F_1} + \vec{F_2} + \vec{F_3} + \cdots$ é o vetor soma de todas as forças exercidas sobre o corpo. O vetor aceleração $\vec{a}$ tem a mesma orientação que o vetor força resultante $\vec{F_{res}}$.
\end{definicao}
	  
Partindo desta definição, podemos encontrar a expressão comumente usada para expressar a 2ª Lei de Newton:
\begin{equation}
	\label{f=ma}
	\vec{F_{res}} = m\vec{a}
\end{equation}

Portanto, temos que a força resultante é igual a massa\footnote{"Massa: a quantidade de matéria num objeto. É também a medida da inércia ou lentidão com que um objeto responde a qualquer esforço feito para movê-lo, pará-lo ou alterar de algum modo o seu estado de movimento."\cite[p.85]{fisica_conceitual}} vezes aceleração. Segue então que, a unidade de força $-$ adotada pelo Sistema Internacional (SI) de Unidades $-$ é dada em newton (1 N = 1 kg $\cdot$ m/s$^{2}$).

Atualmente, a expressão (\ref{f=ma}) é a mais utilizada ao referir-se à 2ª lei de Newton. Contudo, esta não trata-se da formulação original realizada por Newton. Em seu livro, \textit{Princípios Matemáticos de Filosofia Natural}, ele diz "A mudança de movimento é proporcional à força motora imprimida, e é produzida na direção da linha reta na qual aquela força é imprimida."\cite{principios}

A sua formulação original foi realizada em termos da quantidade de movimento, também denominado como momento linear, ou simplesmente momento. Desta forma, o momento ($\vec{p}$) é definido como sendo o produto da massa vezes a velocidade. 
$$\vec{p} = m\vec{v}$$

Assim como a força resultante, a velocidade e o momento são vetores que possuem a mesma orientação. Note que, aplicando a taxa de variação temporal em $\vec{p}$ obtemos:
$$\dfrac{d\vec{p}}{dt} = \dfrac{d[m\vec{v}]}{dt}$$
$$\dfrac{d\vec{p}}{dt} = m\dfrac{d\vec{v}}{dt}$$

Ora, mas como já foi mostrado anteriormente, a taxa de variação temporal da velocidade é exatamente a aceleração, ou seja, $\dfrac{d\vec{v}}{dt} = \vec{a}$. Portanto, substituindo esta informação temos:
$$\dfrac{d\vec{p}}{dt} = m\vec{a}$$

\begin{citacao}
	A força resultante (soma vetorial de todas as forças) que atua sobre uma partícula é dada pela derivada do momento linear da partícula em relação ao tempo. Foi esta forma, e não $\varSigma\vec{F} = m\vec{a}$, que Newton usou no enunciado de sua segunda lei (embora ele chamasse o momento linear de 'quantidade de movimento')\cite[p. 248]{fisica_1}.
\end{citacao}

Logo, pode-se concluir que a taxa de variação temporal do momento é igual à força resultante, sendo esta a formulação correspondente realizada por Newton.
\begin{equation}
	\label{derivada_p}
	\dfrac{d\vec{p}}{dt} = \vec{F_{res}}
\end{equation}

Apesar de ambas (\ref{f=ma} e \ref{derivada_p}) serem equivalentes, existem vantagens em expressar a força resultante como sendo a taxa de variação temporal do momento. Dentre essas vantagens, considere a título de exemplo, a sua validação na Física relativística\footnote{Segundo \citeonline[p.1171]{uma_abordagem4}, se considerássemos um foguete que nunca fica sem combustível, sujeito a uma força constante, e se Física newtoniana fosse correta, o foguete iria se mover cada vez mais rápido e a sua velocidade aumentaria sem limites. Contudo, nada ultrapassa a velocidade da luz. Deste modo, o resultado relativístico mostra que o módulo da velocidade tende a velocidade da luz.}.  

\begin{exemplo}
	Considere um veículo de 1200 Kg fazendo uma trajetória não retilínea em velocidade constante. Ao chegar em um determinado local deste trajeto, o motorista do veículo entra em um trecho retilíneo. A partir disso, ele acelera este veículo até ele alcançar uma velocidade de 30 m/s. Sendo assim, a função horária da posição deste veículo é dada por $S(t) = 10t + t^{2}$ considerando a distância inicial como sendo zero a partir do momento em que o veículo entra em movimento retilíneo e começa a ser acelerado. De acordo com esses dados, calcule em termos do momento linear, a força necessária para proporcionar a aceleração deste veículo.
\end{exemplo}

\begin{solution}
	Como já foi visto, o momento linear é definido como $\vec{p} = m\vec{v}$. Portanto, como temos a massa e a função horária da posição do MRUV do veículo, podemos encontrar a força utilizando estes dados. Para isso, precisamos apenas encontrar a função horária da velocidade. Sendo assim, podemos derivar $S(t)$:	
	$$S(t) = 10t + t^{2}$$
	$$\dfrac{dS}{dt} = \dfrac{d}{dt}[10t + t^{2}]$$
	$$\dfrac{dS}{dt} = \dfrac{d}{dt}[10t] + \dfrac{d}{dt}[t^{2}]$$
	$$\dfrac{dS}{dt} = 10 + 2t$$
	
	Portanto, assim obtemos a função horária da velocidade. Agora, utilizando a definição de momento linear temos:
	$$\vec{p} = 1200(10 + 2t)$$
	
	Por fim, para encontrarmos a força, é suficiente derivar $\vec{p}$.
	$$\dfrac{d\vec{p}}{dt} = \dfrac{d}{dt}[1200(10 + 2t)]$$
	$$\dfrac{d\vec{p}}{dt} = 1200\dfrac{d}{dt}[10] + 1200\dfrac{d}{dt}[2t]$$
	$$\dfrac{d\vec{p}}{dt} = 1200 \cdot 0 + 1200 \cdot 2$$
	$$\dfrac{d\vec{p}}{dt} = 2400 \ \text{N}$$
	
\end{solution}

\begin{exemplo}
	Considere um avião de 378 toneladas que necessite de uma velocidade igual a 216 Km/h para fazer uma decolagem em movimento retilíneo uniformemente variado. Ao cronometrar o tempo desta decolagem, foi obtido um instante de 15s para o avião ter uma velocidade de 72 Km/h e 45s para fazer a decolagem. A partir destes dados, calcule a força necessária para proporcionar a aceleração suficiente para que esta situação ocorra.
\end{exemplo}

\begin{solution}
	De modo análogo ao exemplo anterior, para encontrarmos esta força necessitamos da função horária da velocidade. A partir dos dados fornecidos, podemos construir um sistema linear. Contudo, note que a velocidade é dada em Km/h e o tempo em segundos. Sendo assim, convertendo a velocidade em m/s temos 72 Km/h = 20 m/s e 216 Km/h = 60 m/s. Portanto, temos o sistema linear:
	
	$\begin{cases}
		45a + b = 60 \ (I)\\
		15a + b = 20 \ (II)\\
	\end{cases}$
	$$(II) \ b = 20 - 15a$$ 
	Substituindo $II$ em $(I)$:
	$$(I) \ 45a + 20 - 15a = 60$$
	$$(I) \ 30a = 60 - 20$$
	$$(I) \ 30a = 40$$
	$$(I) \ a = \dfrac{40}{30}$$
	$$(I) \ a = 1,333...$$
	Substituindo agora $I$ em $II$:
	$$(II) \ b = 20 - 15 \cdot 1,333$$ 
	$$(II) \ b = 20 - 20$$
	$$(II) \ b = 0$$
	Portanto, temos $v(t) = 1,333t$ e utilizando a definição de momento linear temos:
	$$\vec{p} = 378(1,333t)$$
	Derivando $\vec{p}$:
	$$\dfrac{d\vec{p}}{dt} = \dfrac{d}{dt}[378(1,333t)]$$
	$$\dfrac{d\vec{p}}{dt} = 378\dfrac{d}{dt}[1,333t]$$
	$$\dfrac{d\vec{p}}{dt} = 378 \cdot 1,333$$
	$$\dfrac{d\vec{p}}{dt} = 504$$
	Note que, a massa do avião foi dada em toneladas e a força é dada em newton, ou seja, kg $\cdot$ m/s$^{2}$. Logo, é necessário multiplicar este resultado por 1000.
	$$\dfrac{d\vec{p}}{dt} = 504 \cdot 1000$$
	$$\dfrac{d\vec{p}}{dt} = 504000 \ \text{N}$$
	
\end{solution}

Nos dois exemplos anteriores, encontramos a força necessária para proporcionar algumas acelerações em certas situações. Contudo, não as encontramos a partir da definição atualmente utilizada para a 2° Lei de Newton, mas sim a formulação original. Como a derivada do momento linear é igual a força resultante, foi possível encontrar assim as forças necessárias para solucionar os exemplos anteriores. 

Vimos inicialmente, neste Capítulo, a definição de velocidade e aceleração que são importantes para definirmos a velocidade e aceleração instantânea, que são taxas de variações temporais. Deste modo, foi possível apresentar algumas aplicações dessas taxas de variações na Física.
% ---
% Considerações Finais
% ---
\chapter*[CONSIDERAÇÕES FINAIS]{CONSIDERAÇÕES FINAIS}
\addcontentsline{toc}{chapter}{CONSIDERAÇÕES FINAIS}
\thispagestyle{empty}
% ---
Tendo em vista uma maior afinidade com o tema desenvolvido, desde a sua exposição na graduação, foi tomado como objetivo principal mostrar algumas aplicações a respeito do Cálculo Diferencial $-$ considerando a Física como objeto de estudo $-$ visando assim ressaltar a sua importância. Seguindo esta linha de raciocínio, uma trajetória foi construída com a finalidade de percorrer alguns objetivos específicos necessários para concluir o objetivo geral proposto para este trabalho.

Com relação aos objetivos específicos, primeiramente focamos em um breve relato histórico. Esta abordagem em suma é importante para narrar um contexto histórico, permeando sobre as grandes mentes responsáveis pelo desenvolvimento desta magnífica descoberta que foi o Cálculo, dando-lhes assim os devidos créditos por isso.

No tocante ao Cálculo Diferencial, abordamos de forma introdutória algumas definições necessárias para a compreensão do mesmo. É conveniente observar que a sua introdução é muito importante, já que as aplicações abordadas fazem referência a ela.

Por fim, algumas aplicações na Física foram expostas, cumprindo o objetivo deste trabalho. Contudo, anteriormente foi introduzida a velocidade e aceleração. É importante frisar em uma abordagem dessas duas grandezas de modo formal, para que o leitor possa compreender ambas de forma mais ampla. Isso se deve por conta das aplicações serem relacionadas a estes dois conceitos da Física.

Realizamos nesse trabalho um estudo introdutório a respeito do Cálculo Diferencial e algumas de suas aplicações na Física. Vale ressaltar que este estudo pode ser ampliado para uma abordagem mais detalhada em relação ao Cálculo Diferencial e principalmente em outras de suas aplicações na Física, que demandam um estudo aprofundado desta área. 

Deste modo, vimos que existem muitas aplicações da velocidade e aceleração instantâneas presentes na Física. Neste trabalho, estão introduzidas algumas aplicações simples destas duas grandezas. Contudo, elas são importantes em determinadas situações que necessitam destes resultados. Sendo assim, esperamos que as informações presentes neste trabalho possam contribuir e levar o leitor a estudos mais aprofundados sobre o respectivo tema.

% ----------------------------------------------------------
% ELEMENTOS PÓS-TEXTUAIS
% ----------------------------------------------------------
%\postextual
% ----------------------------------------------------------

% ----------------------------------------------------------
% Referências bibliográficas
% ----------------------------------------------------------

\bibliography{referencia}

% ----------------------------------------------------------
% Glossário
% ----------------------------------------------------------
%
% Consulte o manual da classe abntex2 para orientações sobre o glossário.
%
%\glossary

%---------------------------------------------------------------------
% INDICE REMISSIVO
%---------------------------------------------------------------------
\phantompart
\printindex[remissivo]
%---------------------------------------------------------------------

\end{document}